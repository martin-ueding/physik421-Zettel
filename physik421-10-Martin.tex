 % Copyright © 2012-2013 Martin Ueding <dev@martin-ueding.de>
%
\input{header.tex}

\usepackage{cancel}
\usepackage{tikz}
\usepackage{pdfpages}

\newcommand{\themodul}{physik421}
\newcommand{\thegruppe}{Gruppe 4 -- Franz Niecknig}
\newcommand{\theuebung}{10}

\ifoot{\footnotesize{Martin Ueding}}
\ihead{\themodul{} -- Übung \theuebung}
\ofoot{\footnotesize{\thegruppe}}

\def\thesection{H \arabic{section}}
\def\thesubsection{\thesection\alph{subsection}}

\setcounter{section}{15}

\title{\themodul{} -- Übung \theuebung}
\subtitle{\thegruppe}
\author{
	Martin Ueding \footnote{\href{mailto:mu@uni-bonn.de}{mu@uni-bonn.de}}
}

\hypersetup{
	pdftitle={\themodul{} - Übung \theuebung},
}

\begin{document}

\maketitle

\begin{center}
	\ccbysadetitle
\end{center}

%%%%%%%%%%%%%%%%%%%%%%%%%%%%%%%%%%%%%%%%%%%%%%%%%%%%%%%%%%%%%%%%%%%%%%%%%%%%%%%
%                                Eichinvarianz                                %
%%%%%%%%%%%%%%%%%%%%%%%%%%%%%%%%%%%%%%%%%%%%%%%%%%%%%%%%%%%%%%%%%%%%%%%%%%%%%%%

\section{Eichinvarianz}

\subsection{Forminvarianz der Schrödingergleichung}
\subsection{Kontinuitätsgleichung}
\subsection{Stromdichte}

%%%%%%%%%%%%%%%%%%%%%%%%%%%%%%%%%%%%%%%%%%%%%%%%%%%%%%%%%%%%%%%%%%%%%%%%%%%%%%%
%                               Landau-Niveaus                                %
%%%%%%%%%%%%%%%%%%%%%%%%%%%%%%%%%%%%%%%%%%%%%%%%%%%%%%%%%%%%%%%%%%%%%%%%%%%%%%%

\section{Landau-Niveaus}

\subsection{Kanonischer Impuls}

Die kanonischen Gleichungen aus der klassichen Mechanik besagen für den Ort:
\[
	\dot x^i = \dpd H{\pi_i} = \frac{\pi_i}m = \frac{p_i}m + \frac q{cm} A^i
\]

Die leite ich nach der Zeit ab und multipliziere mit $m$. Außerdem wende ich
an, dass $\vec E = - \frac 1c \dot{\vec A}$:
\[
	m \ddot x^i = \dot p_i - q E^i
\]

Die ist elektrische Kraft auf ein sonst freies Teilchen.

\subsection{Coulombeichung}

Die Coulomb-Eichung ist (Summenkonvention):
\[
	\partial_i A^i = 0
\]

Mit
\[
	\pi_i = p_i + \frac qc A^i
	= - \ii \hbar \dpd{}{{x_i}} + \frac qc A^i
\]
kann ich den Kommutator schreiben:
\begin{align*}
	[\pi_i, \pi_j]
	&= \sbr{-\ii \hbar \dpd{}{{x_i}} + \frac qc A^i, -\ii \hbar \dpd{}{{x_j}} + \frac qc A^j} \\
	\intertext{%
		Die Ableitungen kommutieren untereinander, die $A$ auch, allerdings
		nicht gemischt.
	}
	&= \sbr{-\ii \hbar \dpd{}{{x_i}}, \frac qc A^j} + \sbr{\frac qc A^i, -\ii \hbar \dpd{}{{x_j}}} \\
	&= -\ii \hbar \frac qc \sbr{\dpd{}{{x_i}}, A^j} + \sbr{A^i, \dpd{}{{x_j}}} \\
	&= - \ii \hbar \frac qc \del{\dpd{}{{x_i}} A^j - \dpd{}{{x_j}} A^i} \\
	&= - \ii \hbar \frac qc \epsilon^i{}_j{}^k \partial_i A^j \\
	&= - \ii \hbar \frac qc B^k
\end{align*}

Wegen $B^1 = B^2 = 0$ ist $[\pi_i, \pi_z] = 0$.

Die Lorentzkraft wirkt nicht in $z$-Richtung, so dass dort ebene Wellen
herauskommen. Daher kann der Separationsansatz so gewählt werden.

\subsection{Landaueichung}
\subsection{Energieniveaus}

%%%%%%%%%%%%%%%%%%%%%%%%%%%%%%%%%%%%%%%%%%%%%%%%%%%%%%%%%%%%%%%%%%%%%%%%%%%%%%%
%                                    Ende                                     %
%%%%%%%%%%%%%%%%%%%%%%%%%%%%%%%%%%%%%%%%%%%%%%%%%%%%%%%%%%%%%%%%%%%%%%%%%%%%%%%

\IfFileExists{\bibliographyfile}{
	%\bibliography{\bibliographyfile}
}{}

\end{document}

% vim: spell spelllang=de
