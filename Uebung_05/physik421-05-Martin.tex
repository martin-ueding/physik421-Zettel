 % Copyright © 2012-2013 Martin Ueding <dev@martin-ueding.de>
%
\input{header.tex}

\usepackage{cancel}
\usepackage{tikz}

\newcommand{\themodul}{physik421}
\newcommand{\thegruppe}{Gruppe 4 -- Franz Niecknig}
\newcommand{\theuebung}{5}

\ifoot{\footnotesize{Martin Ueding}}
\ihead{\themodul{} -- Übung \theuebung}
\ofoot{\footnotesize{\thegruppe}}

\def\thesection{H \arabic{section}}
\def\thesubsection{\thesection\alph{subsection}}

\setcounter{section}{5}

\title{\themodul{} -- Übung \theuebung}
\subtitle{\thegruppe}
\author{
	Martin Ueding \footnote{\href{mailto:mu@uni-bonn.de}{mu@uni-bonn.de}}
}

\hypersetup{
	pdftitle={\themodul{} - Übung \theuebung},
}

\begin{document}

\maketitle

\begin{center}
	\ccbysadetitle
\end{center}

\begin{Form}
\begin{table}[h]
	\centering
	\begin{tabular}{l|c|c|c}
		Aufgabe
		& \ref 1
		& \ref 2
		& $\sum$ \\
		\hline
		Punkte
		& \TextField[name=aufgabe1, width=1cm]{} / 10
		& \TextField[name=aufgabe2, width=1cm]{} / 10
		& \TextField[name=aufgabe3, width=1cm]{} / 20
	\end{tabular}
\end{table}
\end{Form}

%%%%%%%%%%%%%%%%%%%%%%%%%%%%%%%%%%%%%%%%%%%%%%%%%%%%%%%%%%%%%%%%%%%%%%%%%%%%%%%
%               Unschärferelation beim Harmonischen Oszillator               %
%%%%%%%%%%%%%%%%%%%%%%%%%%%%%%%%%%%%%%%%%%%%%%%%%%%%%%%%%%%%%%%%%%%%%%%%%%%%%%%

\section{Unschärferelation beim Harmonischen Oszillator}
\label 1

\subsection{Erwartungswerte}

\begin{align*}
	\braopket n{\hat x}n
	&= \sqrt{\frac{\hbar}{2m\omega}} \braopket n{\hat a + \hat a^\dagger}n \\
	&= \sqrt{\frac{\hbar}{2m\omega}} \del{\braopket n{\hat a}n + \braopket n{\hat a^\dagger}n} \\
	&= \sqrt{\frac{\hbar}{2m\omega}} \del{\sqrt n \braket n{n-1} + \sqrt{n+1} \braket n{n+1}} \\
	&= 0
\end{align*}

Analog: $\braopket n{\hat p}n = 0$.

\begin{align*}
	\braopket n{\hat x^2}n
	&= \frac{\hbar}{2m\omega} \braopket n{\del{\hat a + \hat a^\dagger}^2}n \\
	&= \frac{\hbar}{2m\omega} \braopket n{\hat a^2 + \hat a \hat a^\dagger + \hat a^\dagger \hat a + \hat a^{\dagger2}}n \\
	&= \frac{\hbar}{2m\omega} (2n+1)
\end{align*}

\begin{align*}
	\braopket n{\hat p^2}n
	&= - \frac{\hbar m \omega}{2} \braopket n{\del{\hat a - \hat a^\dagger}^2}n \\
	&= - \frac{\hbar m \omega}{2} \braopket n{\hat a^2 - \hat a \hat a^\dagger - \hat a^\dagger \hat a + \hat a^{\dagger2}}n \\
	&= \hbar m\omega \del{n+\half}
\end{align*}

\subsection{Unschärferelation}

\begin{align*}
	\del{\Deltaup x}^2 \del{\Deltaup p}^2
	&= \del{\bracket{x^2} - \bracket x^2} \del{\bracket{p^2} - \bracket p^2} \\
	&= \hbar ^2 \del{n+\half}^2 \\
	\del{\Deltaup x} \del{\Deltaup p}
	&= \hbar \del{n+\half}
\end{align*}

%%%%%%%%%%%%%%%%%%%%%%%%%%%%%%%%%%%%%%%%%%%%%%%%%%%%%%%%%%%%%%%%%%%%%%%%%%%%%%%
%                                Wellenpakete                                 %
%%%%%%%%%%%%%%%%%%%%%%%%%%%%%%%%%%%%%%%%%%%%%%%%%%%%%%%%%%%%%%%%%%%%%%%%%%%%%%%

\section{Wellenpakete}
\label 2

\subsection{Zeichnung}

\begin{center}
	\begin{tikzpicture}
		\draw[range=-5:5, samples=100, thick] plot (\x, {exp(-(\x+4)^2/.1)});
		\draw[color=blue] (-5, 0) -- (-1, 0) -- (-1, -1) -- (1, -1) -- (1, 0) -- (5, 0);
	\end{tikzpicture}
\end{center}

Wegen $\sigma \ll \abs{x_0 + L/2}$ ist $x = -L/2$ bei mindestens $10 \sigma$,
Integral vernachlässigbar.

\subsection{Entwicklungskoeffizienten}

\begin{align*}
	\braket{\Psi\del\cdot}{\Psi\del{\cdot, 0}}
	&= \braket{\alpha_+ \exp\del{\ii k x } + \alpha_- \exp\del{-\ii kx}}{\frac{1}{\sqrt{\piup \sigma^2}} \exp\del{-\frac{\del{x-x_0}^2}{2\sigma^2}} \exp\del{\ii k_0x}} \\
	&= \braket{\alpha_+ \exp\del{\ii k x }}{\frac{1}{\sqrt{\piup \sigma^2}} \exp\del{-\frac{\del{x-x_0}^2}{2\sigma^2}} \exp\del{\ii k_0x}} \\
	&\phantom= + \braket{\alpha_- \exp\del{-\ii kx}}{\frac{1}{\sqrt{\piup \sigma^2}} \exp\del{-\frac{\del{x-x_0}^2}{2\sigma^2}} \exp\del{\ii k_0x}} \\
	\displaybreak[0]
	%
	&= \braket{\alpha_+ }{\exp\del{-\ii k x} \frac{1}{\sqrt{\piup \sigma^2}} \exp\del{-\frac{\del{x-x_0}^2}{2\sigma^2}} \exp\del{\ii k_0x}} \\
	&\phantom= + \braket{\alpha_-}{\exp\del{\ii kx} \frac{1}{\sqrt{\piup \sigma^2}} \exp\del{-\frac{\del{x-x_0}^2}{2\sigma^2}} \exp\del{\ii k_0x}} \\
	\displaybreak[0]
	%
	&= \alpha_+^\dagger \int_{-\infty}^\infty \dif x \,
\exp\del{-\ii k x} \frac{1}{\sqrt{\piup \sigma^2}} \exp\del{-\frac{\del{x-x_0}^2}{2\sigma^2}} \exp\del{\ii k_0x} \\
	&\phantom= + \alpha_-^\dagger \int_{-\infty}^\infty \dif x \,
	\displaybreak[0]
\exp\del{\ii kx} \frac{1}{\sqrt{\piup \sigma^2}} \exp\del{-\frac{\del{x-x_0}^2}{2\sigma^2}} \exp\del{\ii k_0x} \\
%
&= \alpha_+^\dagger \fourier\sbr{\Psi}(k) + \alpha_-^\dagger \fourier^{-1}\sbr{\Psi}(k) \\
	\displaybreak[0]
&= \alpha_+^\dagger \fourier\sbr{\Psi}(k) + \alpha_-^\dagger \fourier\sbr{\Psi}(-k) \\
	\displaybreak[0]
&= \alpha_+^\dagger \tilde{\Psi}(k) + \alpha_-^\dagger \tilde{\Psi}(-k)
\end{align*}

\subsection{Fouriertransformation}

\begin{align*}
	\tilde\Psi(k)
	&= \frac{1}{\sqrt{\sigma^2\piup}} \int_{-\infty}^\infty \dif x \,
	\eup^{- \ii k x} \exp\del{-\frac{\del{x-x_0}^2}{2\sigma^2}} \eup^{\ii k_0 x} \\
	\displaybreak[0]
	%
	&= \frac{1}{\sqrt{\sigma^2\piup}} \int_{-\infty}^\infty \dif x \,
	\exp\del{-\frac{\del{x-x_0}^2}{2\sigma^2} + \ii k_0 x- \ii k x} \\
	\displaybreak[0]
	%
	&= \frac{1}{\sqrt{\sigma^2\piup}} \int_{-\infty}^\infty \dif x \,
	\exp\del{-\frac{\del{x-x_0}^2}{2\sigma^2} + \ii \del{k_0-k} x} \\
	\displaybreak[0]
	%
	&= \frac{1}{\sqrt{\sigma^2\piup}} \int_{-\infty}^\infty \dif x \,
	\exp\del{-\frac{x^2}{2\sigma^2} + \frac{xx_0}{\sigma^2} -\frac{x_0^2}{2\sigma^2}+ \ii \del{k_0-k} x} \\
	\displaybreak[0]
	%
	&= \frac{1}{\sqrt{\sigma^2\piup}} \int_{-\infty}^\infty \dif x \,
	\exp\del{
		-\del{\frac{x}{\sqrt{2\sigma^2}} - \frac{\sqrt{2\sigma^2}}2 \del{ \frac{x_0}{2\sigma^2} + \ii \del{k_0-k}}}^2 - \frac{x_0^2}{2\sigma^2} + \frac{2\sigma^2}4 \del{\frac{x_0}{\sigma^2} + \ii \del{k_0-k}}^2
		} \\
	%
		\intertext{%
			\[
				\xi := \frac{x}{\sqrt{2\sigma^2}} - \frac{\sqrt{2\sigma^2}}2 \del{ \frac{x_0}{2\sigma^2} + \ii \del{k_0-k}}
			\]
		}
	%
	&= \frac{\sqrt 2}{\sqrt{\piup}} 
	\exp\del{
		- \frac{x_0^2}{2\sigma^2} + \frac{2\sigma^2}4 \del{\frac{x_0}{\sigma^2} + \ii \del{k_0-k}}^2
		}
		\int_{-\infty}^\infty \dif \xi \, \eup^{-\xi^2} \\
	\displaybreak[0]
	%
	&= \sqrt 2 \exp\del{
		-\half \del{k - k_0} \del{2 \ii x_0 + \sigma^2 \del{k - k_0}}
		}
\end{align*}

\subsection{Wert von $\alpha_+$}

\begin{align*}
	\Psi(x, 0)
	&= \frac1{2\piup} \int_0^\infty \dif k \,
	\alpha_+^* \tilde \Psi(k, 0) \del{\alpha_+ \ee^{\ii kx} + \alpha_- \ee^{-\ii kx}} \\
	%
	&= \frac1{2\piup} \int_0^\infty \dif k \,
	\abs{\alpha_+}^2 \tilde \Psi(k, 0) \ee^{\ii kx}
	+ \frac1{2\piup} \int_0^\infty \dif k \,
	\alpha_+^* \alpha_- \tilde \Psi(k, 0) \ee^{-\ii kx} \\
	%
	&= \abs{\alpha_+}^2 \psi(x, 0) + \underbrace{\frac1{2\piup} \alpha_+^* \alpha_- \psi(-x, 0)}_{=0}
\end{align*}

$\abs{\alpha_+} = 1$.

\subsection{Gruppengeschwindigkeit}

\[
	v_\text G = \dpd \omega k \del{k_0}
	= \frac\hbar m k_0
\]

Große $t$: Transmittierte Welle nach rechts, reflektierte nach links.

%%%%%%%%%%%%%%%%%%%%%%%%%%%%%%%%%%%%%%%%%%%%%%%%%%%%%%%%%%%%%%%%%%%%%%%%%%%%%%%
%                                    Ende                                     %
%%%%%%%%%%%%%%%%%%%%%%%%%%%%%%%%%%%%%%%%%%%%%%%%%%%%%%%%%%%%%%%%%%%%%%%%%%%%%%%

\IfFileExists{\bibliographyfile}{
	%\bibliography{\bibliographyfile}
}{}

\end{document}

% vim: spell spelllang=de
