% Copyright © 2012-2013 Martin Ueding <dev@martin-ueding.de>
%
\input{header.tex}

\usepackage{cancel}
\usepackage{tikz}

\newcommand{\themodul}{physik421}
\newcommand{\thegruppe}{Gruppe 4 -- Franz Niecknig}
\newcommand{\theuebung}{4}

\ifoot{\footnotesize{Martin Ueding}}
\ihead{\themodul{} -- Übung \theuebung}
\ofoot{\footnotesize{\thegruppe}}

\def\thesection{H \arabic{section}}
\def\thesubsection{\thesection\alph{subsection}}

\setcounter{section}{4}

\title{\themodul{} -- Übung \theuebung}
\subtitle{\thegruppe}
\author{
	Martin Ueding \footnote{\href{mailto:mu@uni-bonn.de}{mu@uni-bonn.de}}
}

\hypersetup{
	pdftitle={\themodul{} - Übung \theuebung},
}

\begin{document}

\maketitle

\begin{center}
	\ccbysadetitle
\end{center}

\begin{Form}
\begin{table}[h]
	\centering
	\begin{tabular}{l|c}
		Aufgabe
		& \ref 1 \\
		\hline
		Punkte
		& \TextField[name=aufgabe1, width=1cm]{} / 20
	\end{tabular}
\end{table}
\end{Form}

%%%%%%%%%%%%%%%%%%%%%%%%%%%%%%%%%%%%%%%%%%%%%%%%%%%%%%%%%%%%%%%%%%%%%%%%%%%%%%%
%                            Hermetische Polynome                             %
%%%%%%%%%%%%%%%%%%%%%%%%%%%%%%%%%%%%%%%%%%%%%%%%%%%%%%%%%%%%%%%%%%%%%%%%%%%%%%%

\section{Hermetische Polynome}
\label 1

\subsection{Ableitung}
\label{1a}

Diesen Ansatz habe ich aus \cite{koeppel-hermite_beiblatt}. Zuerst betrachte
ich eine Funktion $f(x) = \ee^{-x^2}$, deren Ableitungen fast die Hermetischen
Polynome ergeben:
\begin{align*}
	H_n(z)
	&= \del-^n \ee^{x^2} \dod[n]{}x \ee^{-x^2} \\
	&= \del-^n \ee^{x^2} \dod[n]{}x f(x) \\
	&= \del-^n \ee^{x^2} f^{(n)}(x)
\end{align*}

Dann betrachte ich die Taylorentwicklung der Funktion $f(x - y)$ um den Punkt
$y = 0$:
\begin{align*}
	f(x - y)
	&= \ee^{-(x-y)^2} \\
	&= \sum_{n=0}^\infty \del-^n \frac{f^{(n)}(x)}{n!} y^n \\
	\intertext{%
		Setze die obige Relation ein.
	}
	&= \sum_{n=0}^\infty \frac{y^n}{n!} H_n(x) \ee^{-x^2} \\
	f(x-y) \ee^{x^2} 
	&= \sum_{n=0}^\infty \frac{y^n}{n!} H_n(x)
\end{align*}

Die linke Seite ist die erzeugende Funktion, ich nenne sie $w(x, y)$:
\[
	w(x, y) = \ee^{2xy-y^2}
\]

Die partielle Ableitung nach $x$ liefert:
\[
	\dpd{w(x, y)}x = 2 y w(x, y)
	= \sum_{n=0}^\infty \frac{y^n}{n!} \dod{}x H_n(z)
\]

Allerdings kann ich den Ausdruck $2 y w(x, y)$ auch erhalten, in dem ich
folgendes rechne:
\begin{align*}
	2 w(x, y)
	&= 2 \sum_{n=0}^\infty \frac{y^n}{n!} H_n(x) \\
	2 y w(x, y)
	&= 2 \sum_{n=0}^\infty \frac{y^{n+1}}{n!} H_n(x) \\
	\intertext{%
		Verschiebe die Indizes.
	}
	&= 2 \sum_{n=0}^\infty \frac{y^{n}}{(n-1)!} H_{n-1}(x) \\
	&= 2 \sum_{n=0}^\infty \frac{y^{n}}{n!} n H_{n-1}(x)
\end{align*}

Ein Koeffizientenvergleich der beiden Ausdrücke liefert die gesuchte Relation:
\[
	\dpd{}x H_n(x) = 2n H_{n-1}(x)
\]

\subsection{Orthogonalität}
\subsection{Eigenfunktionen}
\subsection{Rekursionsbeziehung}

\begin{align*}
	\dod{}x H_n(x)
	&= \del-^n \dod{}x \ee^{x^2} \dod[n]{}x \ee^{-x^2} \\
	&= \del-^n 2x \ee^{x^2} \dod[n]{}x \ee^{-x^2} + \del-^n \ee^{x^2} \dod[n+1]{}x \ee^{-x^2} \\
	&= 2x H_n(x) - H_{n+1}(x) \\
	\intertext{%
		Wende die Relation aus Aufgabe \ref{1a} an.
	}
	2n H_{n-1}(x)
	&= 2x H_n(x) - H_{n+1}(x) \\
	H_{n+1}(x)
	&= 2x H_n(x) - 2n H_{n-1}(x)
\end{align*}

\subsection{Differentialgleichung}

%%%%%%%%%%%%%%%%%%%%%%%%%%%%%%%%%%%%%%%%%%%%%%%%%%%%%%%%%%%%%%%%%%%%%%%%%%%%%%%
%                                    Ende                                     %
%%%%%%%%%%%%%%%%%%%%%%%%%%%%%%%%%%%%%%%%%%%%%%%%%%%%%%%%%%%%%%%%%%%%%%%%%%%%%%%

\IfFileExists{\bibliographyfile}{
	\bibliography{\bibliographyfile}
}{}

\end{document}

% vim: spell spelllang=de
