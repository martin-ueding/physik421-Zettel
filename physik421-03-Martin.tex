% Copyright © 2012-2013 Martin Ueding <dev@martin-ueding.de>
%
% Copyright © 2012-2013 Martin Ueding <dev@martin-ueding.de>

% This is my general purpose LaTeX header file for writing German documents.
% Ideally, you include this using a simple ``% Copyright © 2012-2013 Martin Ueding <dev@martin-ueding.de>

% This is my general purpose LaTeX header file for writing German documents.
% Ideally, you include this using a simple ``% Copyright © 2012-2013 Martin Ueding <dev@martin-ueding.de>

% This is my general purpose LaTeX header file for writing German documents.
% Ideally, you include this using a simple ``\input{header.tex}`` in your main
% document and start with ``\title`` and ``\begin{document}`` afterwards.

% If you need to add additional packages, I recommend not doing this in this
% file, but in your main document. That way, you can just drop in a new
% ``header.tex`` and get all the new commands without having to merge manually.

% Since this file encorporates a CC-BY-SA fragment, this whole files is
% licensed under the CC-BY-SA license.

\documentclass[11pt, ngerman, fleqn, DIV=15, headinclude]{scrartcl}

\usepackage{graphicx}

%%%%%%%%%%%%%%%%%%%%%%%%%%%%%%%%%%%%%%%%%%%%%%%%%%%%%%%%%%%%%%%%%%%%%%%%%%%%%%%
%                                Locale, date                                 %
%%%%%%%%%%%%%%%%%%%%%%%%%%%%%%%%%%%%%%%%%%%%%%%%%%%%%%%%%%%%%%%%%%%%%%%%%%%%%%%

\usepackage{babel}
\usepackage[iso]{isodate}

%%%%%%%%%%%%%%%%%%%%%%%%%%%%%%%%%%%%%%%%%%%%%%%%%%%%%%%%%%%%%%%%%%%%%%%%%%%%%%%
%                          Margins and other spacing                          %
%%%%%%%%%%%%%%%%%%%%%%%%%%%%%%%%%%%%%%%%%%%%%%%%%%%%%%%%%%%%%%%%%%%%%%%%%%%%%%%

\usepackage[activate]{pdfcprot}
\usepackage[parfill]{parskip}
\usepackage{setspace}

\setlength{\columnsep}{2cm}

%%%%%%%%%%%%%%%%%%%%%%%%%%%%%%%%%%%%%%%%%%%%%%%%%%%%%%%%%%%%%%%%%%%%%%%%%%%%%%%
%                                    Color                                    %
%%%%%%%%%%%%%%%%%%%%%%%%%%%%%%%%%%%%%%%%%%%%%%%%%%%%%%%%%%%%%%%%%%%%%%%%%%%%%%%

\usepackage{color}

\definecolor{darkblue}{rgb}{0,0,.5}
\definecolor{darkgreen}{rgb}{0,.5,0}
\definecolor{darkred}{rgb}{.7,0,0}

%%%%%%%%%%%%%%%%%%%%%%%%%%%%%%%%%%%%%%%%%%%%%%%%%%%%%%%%%%%%%%%%%%%%%%%%%%%%%%%
%                         Font and font like settings                         %
%%%%%%%%%%%%%%%%%%%%%%%%%%%%%%%%%%%%%%%%%%%%%%%%%%%%%%%%%%%%%%%%%%%%%%%%%%%%%%%

% This replaces all fonts with Bitstream Charter, Bitstream Vera Sans and
% Bitstream Vera Mono. Math will be rendered in Charter.
\usepackage[charter, greekuppercase=italicized]{mathdesign}
\usepackage{beramono}
\usepackage{berasans}

% Bold, sans-serif tensors. This fragment is taken from “egreg” from
% http://tex.stackexchange.com/a/82747/8945 and licensed under `CC-BY-SA
% <https://creativecommons.org/licenses/by-sa/3.0/>`_.
\usepackage{bm}
\DeclareMathAlphabet{\mathsfit}{\encodingdefault}{\sfdefault}{m}{sl}
\SetMathAlphabet{\mathsfit}{bold}{\encodingdefault}{\sfdefault}{bx}{sl}
\newcommand{\tens}[1]{\bm{\mathsfit{#1}}}

% Bold vectors.
\renewcommand{\vec}[1]{\boldsymbol{#1}}

%%%%%%%%%%%%%%%%%%%%%%%%%%%%%%%%%%%%%%%%%%%%%%%%%%%%%%%%%%%%%%%%%%%%%%%%%%%%%%%
%                               Input encoding                                %
%%%%%%%%%%%%%%%%%%%%%%%%%%%%%%%%%%%%%%%%%%%%%%%%%%%%%%%%%%%%%%%%%%%%%%%%%%%%%%%

\usepackage[T1]{fontenc}
\usepackage[utf8]{inputenc}

%%%%%%%%%%%%%%%%%%%%%%%%%%%%%%%%%%%%%%%%%%%%%%%%%%%%%%%%%%%%%%%%%%%%%%%%%%%%%%%
%                         Hyperrefs and PDF metadata                          %
%%%%%%%%%%%%%%%%%%%%%%%%%%%%%%%%%%%%%%%%%%%%%%%%%%%%%%%%%%%%%%%%%%%%%%%%%%%%%%%

\usepackage{hyperref}
\usepackage{lastpage}

% This sets the author in the properties of the PDF as well. If you want to
% change it, just override it with another ``\hypersetup`` call.
\hypersetup{
	breaklinks=false,
	citecolor=darkgreen,
	colorlinks=true,
	linkcolor=black,
	menucolor=black,
	pdfauthor={Martin Ueding},
	urlcolor=darkblue,
}

%%%%%%%%%%%%%%%%%%%%%%%%%%%%%%%%%%%%%%%%%%%%%%%%%%%%%%%%%%%%%%%%%%%%%%%%%%%%%%%
%                               Math Operators                                %
%%%%%%%%%%%%%%%%%%%%%%%%%%%%%%%%%%%%%%%%%%%%%%%%%%%%%%%%%%%%%%%%%%%%%%%%%%%%%%%

% AMS environments like ``align`` and theorems like ``proof``.
\usepackage{amsmath}
\usepackage{amsthm}

% Common math constructs like partial derivatives.
\usepackage{commath}

% Physical units.
\usepackage{siunitx}

% Word like operators.
\DeclareMathOperator{\acosh}{arcosh}
\DeclareMathOperator{\arcosh}{arcosh}
\DeclareMathOperator{\arcsinh}{arsinh}
\DeclareMathOperator{\arsinh}{arsinh}
\DeclareMathOperator{\asinh}{arsinh}
\DeclareMathOperator{\card}{card}
\DeclareMathOperator{\csch}{cshs}
\DeclareMathOperator{\diam}{diam}
\DeclareMathOperator{\sech}{sech}
\renewcommand{\Im}{\mathop{{}\mathrm{Im}}\nolimits}
\renewcommand{\Re}{\mathop{{}\mathrm{Re}}\nolimits}

% Fourier transform.
\DeclareMathOperator{\fourier}{\ensuremath{\mathcal{F}}}

% Roman versions of “e” and “i” to serve as Euler's number and the imaginary
% constant.
\newcommand{\ee}{\eup}
\newcommand{\eup}{\mathrm e}
\newcommand{\ii}{\iup}
\newcommand{\iup}{\mathrm i}

% Symbols for the various mathematical fields (natural numbers, integers,
% rational numbers, real numbers, complex numbers).
\newcommand{\C}{\ensuremath{\mathbb C}}
\newcommand{\N}{\ensuremath{\mathbb N}}
\newcommand{\Q}{\ensuremath{\mathbb Q}}
\newcommand{\R}{\ensuremath{\mathbb R}}
\newcommand{\Z}{\ensuremath{\mathbb Z}}

% Shape like operators.
\DeclareMathOperator{\dalambert}{\Box}
\DeclareMathOperator{\laplace}{\bigtriangleup}
\newcommand{\curl}{\vnabla \times}
\newcommand{\divergence}[1]{\inner{\vnabla}{#1}}
\newcommand{\vnabla}{\vec \nabla}

\newcommand{\half}{\frac 12}

% Unit vector (German „Einheitsvektor“).
\newcommand{\ev}{\hat{\vec e}}

% Scientific notation for large numbers.
\newcommand{\e}[1]{\cdot 10^{#1}}

% Mathematician's notation for the inner (scalar, dot) product.
\newcommand{\inner}[2]{\left\langle #1, #2 \right\rangle}

% Placeholders.
\newcommand{\emesswert}{\del{\messwert \pm \messwert}}
\newcommand{\fehlt}{\textcolor{darkred}{Hier fehlen noch Inhalte.}}
\newcommand{\messwert}{\textcolor{blue}{\square}}
\newcommand{\punkte}{\textcolor{white}{xxxxx}}

% Separator for equations on a single line.
\newcommand{\eqnsep}{,\quad}

% Quantum Mechanics
\newcommand{\bra}[1]{\left\langle #1 \right|}
\newcommand{\ket}[1]{\left| #1 \right\rangle}
\newcommand{\braket}[2]{\left\langle #1 \left. \vphantom{#1 #2} \right| #2 \right\rangle}

%%%%%%%%%%%%%%%%%%%%%%%%%%%%%%%%%%%%%%%%%%%%%%%%%%%%%%%%%%%%%%%%%%%%%%%%%%%%%%%
%                                  Headings                                   %
%%%%%%%%%%%%%%%%%%%%%%%%%%%%%%%%%%%%%%%%%%%%%%%%%%%%%%%%%%%%%%%%%%%%%%%%%%%%%%%

% This will set fancy headings to the top of the page. The page number will be
% accompanied by the total number of pages. That way, you will know if any page
% is missing.
%
% If you do not want this for your document, you can just use
% ``\pagestyle{plain}``.

\usepackage{scrpage2}

\pagestyle{scrheadings}

\automark{section}

\cfoot{\footnotesize{Seite \thepage\ / \pageref{LastPage}}}
\chead{}
\ihead{}
\ohead{\rightmark}

\setheadsepline{.4pt}

%%%%%%%%%%%%%%%%%%%%%%%%%%%%%%%%%%%%%%%%%%%%%%%%%%%%%%%%%%%%%%%%%%%%%%%%%%%%%%%
%                            Bibliography (BibTeX)                            %
%%%%%%%%%%%%%%%%%%%%%%%%%%%%%%%%%%%%%%%%%%%%%%%%%%%%%%%%%%%%%%%%%%%%%%%%%%%%%%%

\newcommand{\bibliographyfile}{../../zentrale_BibTeX/Central}

%%%%%%%%%%%%%%%%%%%%%%%%%%%%%%%%%%%%%%%%%%%%%%%%%%%%%%%%%%%%%%%%%%%%%%%%%%%%%%%
%                                Abbreviations                                %
%%%%%%%%%%%%%%%%%%%%%%%%%%%%%%%%%%%%%%%%%%%%%%%%%%%%%%%%%%%%%%%%%%%%%%%%%%%%%%%

\newcommand{\dhabk}{\mbox{d.\,h.}}
`` in your main
% document and start with ``\title`` and ``\begin{document}`` afterwards.

% If you need to add additional packages, I recommend not doing this in this
% file, but in your main document. That way, you can just drop in a new
% ``header.tex`` and get all the new commands without having to merge manually.

% Since this file encorporates a CC-BY-SA fragment, this whole files is
% licensed under the CC-BY-SA license.

\documentclass[11pt, ngerman, fleqn, DIV=15, headinclude]{scrartcl}

\usepackage{graphicx}

%%%%%%%%%%%%%%%%%%%%%%%%%%%%%%%%%%%%%%%%%%%%%%%%%%%%%%%%%%%%%%%%%%%%%%%%%%%%%%%
%                                Locale, date                                 %
%%%%%%%%%%%%%%%%%%%%%%%%%%%%%%%%%%%%%%%%%%%%%%%%%%%%%%%%%%%%%%%%%%%%%%%%%%%%%%%

\usepackage{babel}
\usepackage[iso]{isodate}

%%%%%%%%%%%%%%%%%%%%%%%%%%%%%%%%%%%%%%%%%%%%%%%%%%%%%%%%%%%%%%%%%%%%%%%%%%%%%%%
%                          Margins and other spacing                          %
%%%%%%%%%%%%%%%%%%%%%%%%%%%%%%%%%%%%%%%%%%%%%%%%%%%%%%%%%%%%%%%%%%%%%%%%%%%%%%%

\usepackage[activate]{pdfcprot}
\usepackage[parfill]{parskip}
\usepackage{setspace}

\setlength{\columnsep}{2cm}

%%%%%%%%%%%%%%%%%%%%%%%%%%%%%%%%%%%%%%%%%%%%%%%%%%%%%%%%%%%%%%%%%%%%%%%%%%%%%%%
%                                    Color                                    %
%%%%%%%%%%%%%%%%%%%%%%%%%%%%%%%%%%%%%%%%%%%%%%%%%%%%%%%%%%%%%%%%%%%%%%%%%%%%%%%

\usepackage{color}

\definecolor{darkblue}{rgb}{0,0,.5}
\definecolor{darkgreen}{rgb}{0,.5,0}
\definecolor{darkred}{rgb}{.7,0,0}

%%%%%%%%%%%%%%%%%%%%%%%%%%%%%%%%%%%%%%%%%%%%%%%%%%%%%%%%%%%%%%%%%%%%%%%%%%%%%%%
%                         Font and font like settings                         %
%%%%%%%%%%%%%%%%%%%%%%%%%%%%%%%%%%%%%%%%%%%%%%%%%%%%%%%%%%%%%%%%%%%%%%%%%%%%%%%

% This replaces all fonts with Bitstream Charter, Bitstream Vera Sans and
% Bitstream Vera Mono. Math will be rendered in Charter.
\usepackage[charter, greekuppercase=italicized]{mathdesign}
\usepackage{beramono}
\usepackage{berasans}

% Bold, sans-serif tensors. This fragment is taken from “egreg” from
% http://tex.stackexchange.com/a/82747/8945 and licensed under `CC-BY-SA
% <https://creativecommons.org/licenses/by-sa/3.0/>`_.
\usepackage{bm}
\DeclareMathAlphabet{\mathsfit}{\encodingdefault}{\sfdefault}{m}{sl}
\SetMathAlphabet{\mathsfit}{bold}{\encodingdefault}{\sfdefault}{bx}{sl}
\newcommand{\tens}[1]{\bm{\mathsfit{#1}}}

% Bold vectors.
\renewcommand{\vec}[1]{\boldsymbol{#1}}

%%%%%%%%%%%%%%%%%%%%%%%%%%%%%%%%%%%%%%%%%%%%%%%%%%%%%%%%%%%%%%%%%%%%%%%%%%%%%%%
%                               Input encoding                                %
%%%%%%%%%%%%%%%%%%%%%%%%%%%%%%%%%%%%%%%%%%%%%%%%%%%%%%%%%%%%%%%%%%%%%%%%%%%%%%%

\usepackage[T1]{fontenc}
\usepackage[utf8]{inputenc}

%%%%%%%%%%%%%%%%%%%%%%%%%%%%%%%%%%%%%%%%%%%%%%%%%%%%%%%%%%%%%%%%%%%%%%%%%%%%%%%
%                         Hyperrefs and PDF metadata                          %
%%%%%%%%%%%%%%%%%%%%%%%%%%%%%%%%%%%%%%%%%%%%%%%%%%%%%%%%%%%%%%%%%%%%%%%%%%%%%%%

\usepackage{hyperref}
\usepackage{lastpage}

% This sets the author in the properties of the PDF as well. If you want to
% change it, just override it with another ``\hypersetup`` call.
\hypersetup{
	breaklinks=false,
	citecolor=darkgreen,
	colorlinks=true,
	linkcolor=black,
	menucolor=black,
	pdfauthor={Martin Ueding},
	urlcolor=darkblue,
}

%%%%%%%%%%%%%%%%%%%%%%%%%%%%%%%%%%%%%%%%%%%%%%%%%%%%%%%%%%%%%%%%%%%%%%%%%%%%%%%
%                               Math Operators                                %
%%%%%%%%%%%%%%%%%%%%%%%%%%%%%%%%%%%%%%%%%%%%%%%%%%%%%%%%%%%%%%%%%%%%%%%%%%%%%%%

% AMS environments like ``align`` and theorems like ``proof``.
\usepackage{amsmath}
\usepackage{amsthm}

% Common math constructs like partial derivatives.
\usepackage{commath}

% Physical units.
\usepackage{siunitx}

% Word like operators.
\DeclareMathOperator{\acosh}{arcosh}
\DeclareMathOperator{\arcosh}{arcosh}
\DeclareMathOperator{\arcsinh}{arsinh}
\DeclareMathOperator{\arsinh}{arsinh}
\DeclareMathOperator{\asinh}{arsinh}
\DeclareMathOperator{\card}{card}
\DeclareMathOperator{\csch}{cshs}
\DeclareMathOperator{\diam}{diam}
\DeclareMathOperator{\sech}{sech}
\renewcommand{\Im}{\mathop{{}\mathrm{Im}}\nolimits}
\renewcommand{\Re}{\mathop{{}\mathrm{Re}}\nolimits}

% Fourier transform.
\DeclareMathOperator{\fourier}{\ensuremath{\mathcal{F}}}

% Roman versions of “e” and “i” to serve as Euler's number and the imaginary
% constant.
\newcommand{\ee}{\eup}
\newcommand{\eup}{\mathrm e}
\newcommand{\ii}{\iup}
\newcommand{\iup}{\mathrm i}

% Symbols for the various mathematical fields (natural numbers, integers,
% rational numbers, real numbers, complex numbers).
\newcommand{\C}{\ensuremath{\mathbb C}}
\newcommand{\N}{\ensuremath{\mathbb N}}
\newcommand{\Q}{\ensuremath{\mathbb Q}}
\newcommand{\R}{\ensuremath{\mathbb R}}
\newcommand{\Z}{\ensuremath{\mathbb Z}}

% Shape like operators.
\DeclareMathOperator{\dalambert}{\Box}
\DeclareMathOperator{\laplace}{\bigtriangleup}
\newcommand{\curl}{\vnabla \times}
\newcommand{\divergence}[1]{\inner{\vnabla}{#1}}
\newcommand{\vnabla}{\vec \nabla}

\newcommand{\half}{\frac 12}

% Unit vector (German „Einheitsvektor“).
\newcommand{\ev}{\hat{\vec e}}

% Scientific notation for large numbers.
\newcommand{\e}[1]{\cdot 10^{#1}}

% Mathematician's notation for the inner (scalar, dot) product.
\newcommand{\inner}[2]{\left\langle #1, #2 \right\rangle}

% Placeholders.
\newcommand{\emesswert}{\del{\messwert \pm \messwert}}
\newcommand{\fehlt}{\textcolor{darkred}{Hier fehlen noch Inhalte.}}
\newcommand{\messwert}{\textcolor{blue}{\square}}
\newcommand{\punkte}{\textcolor{white}{xxxxx}}

% Separator for equations on a single line.
\newcommand{\eqnsep}{,\quad}

% Quantum Mechanics
\newcommand{\bra}[1]{\left\langle #1 \right|}
\newcommand{\ket}[1]{\left| #1 \right\rangle}
\newcommand{\braket}[2]{\left\langle #1 \left. \vphantom{#1 #2} \right| #2 \right\rangle}

%%%%%%%%%%%%%%%%%%%%%%%%%%%%%%%%%%%%%%%%%%%%%%%%%%%%%%%%%%%%%%%%%%%%%%%%%%%%%%%
%                                  Headings                                   %
%%%%%%%%%%%%%%%%%%%%%%%%%%%%%%%%%%%%%%%%%%%%%%%%%%%%%%%%%%%%%%%%%%%%%%%%%%%%%%%

% This will set fancy headings to the top of the page. The page number will be
% accompanied by the total number of pages. That way, you will know if any page
% is missing.
%
% If you do not want this for your document, you can just use
% ``\pagestyle{plain}``.

\usepackage{scrpage2}

\pagestyle{scrheadings}

\automark{section}

\cfoot{\footnotesize{Seite \thepage\ / \pageref{LastPage}}}
\chead{}
\ihead{}
\ohead{\rightmark}

\setheadsepline{.4pt}

%%%%%%%%%%%%%%%%%%%%%%%%%%%%%%%%%%%%%%%%%%%%%%%%%%%%%%%%%%%%%%%%%%%%%%%%%%%%%%%
%                            Bibliography (BibTeX)                            %
%%%%%%%%%%%%%%%%%%%%%%%%%%%%%%%%%%%%%%%%%%%%%%%%%%%%%%%%%%%%%%%%%%%%%%%%%%%%%%%

\newcommand{\bibliographyfile}{../../zentrale_BibTeX/Central}

%%%%%%%%%%%%%%%%%%%%%%%%%%%%%%%%%%%%%%%%%%%%%%%%%%%%%%%%%%%%%%%%%%%%%%%%%%%%%%%
%                                Abbreviations                                %
%%%%%%%%%%%%%%%%%%%%%%%%%%%%%%%%%%%%%%%%%%%%%%%%%%%%%%%%%%%%%%%%%%%%%%%%%%%%%%%

\newcommand{\dhabk}{\mbox{d.\,h.}}
`` in your main
% document and start with ``\title`` and ``\begin{document}`` afterwards.

% If you need to add additional packages, I recommend not doing this in this
% file, but in your main document. That way, you can just drop in a new
% ``header.tex`` and get all the new commands without having to merge manually.

% Since this file encorporates a CC-BY-SA fragment, this whole files is
% licensed under the CC-BY-SA license.

\documentclass[11pt, ngerman, fleqn, DIV=15, headinclude]{scrartcl}

\usepackage{graphicx}

%%%%%%%%%%%%%%%%%%%%%%%%%%%%%%%%%%%%%%%%%%%%%%%%%%%%%%%%%%%%%%%%%%%%%%%%%%%%%%%
%                                Locale, date                                 %
%%%%%%%%%%%%%%%%%%%%%%%%%%%%%%%%%%%%%%%%%%%%%%%%%%%%%%%%%%%%%%%%%%%%%%%%%%%%%%%

\usepackage{babel}
\usepackage[iso]{isodate}

%%%%%%%%%%%%%%%%%%%%%%%%%%%%%%%%%%%%%%%%%%%%%%%%%%%%%%%%%%%%%%%%%%%%%%%%%%%%%%%
%                          Margins and other spacing                          %
%%%%%%%%%%%%%%%%%%%%%%%%%%%%%%%%%%%%%%%%%%%%%%%%%%%%%%%%%%%%%%%%%%%%%%%%%%%%%%%

\usepackage[activate]{pdfcprot}
\usepackage[parfill]{parskip}
\usepackage{setspace}

\setlength{\columnsep}{2cm}

%%%%%%%%%%%%%%%%%%%%%%%%%%%%%%%%%%%%%%%%%%%%%%%%%%%%%%%%%%%%%%%%%%%%%%%%%%%%%%%
%                                    Color                                    %
%%%%%%%%%%%%%%%%%%%%%%%%%%%%%%%%%%%%%%%%%%%%%%%%%%%%%%%%%%%%%%%%%%%%%%%%%%%%%%%

\usepackage{color}

\definecolor{darkblue}{rgb}{0,0,.5}
\definecolor{darkgreen}{rgb}{0,.5,0}
\definecolor{darkred}{rgb}{.7,0,0}

%%%%%%%%%%%%%%%%%%%%%%%%%%%%%%%%%%%%%%%%%%%%%%%%%%%%%%%%%%%%%%%%%%%%%%%%%%%%%%%
%                         Font and font like settings                         %
%%%%%%%%%%%%%%%%%%%%%%%%%%%%%%%%%%%%%%%%%%%%%%%%%%%%%%%%%%%%%%%%%%%%%%%%%%%%%%%

% This replaces all fonts with Bitstream Charter, Bitstream Vera Sans and
% Bitstream Vera Mono. Math will be rendered in Charter.
\usepackage[charter, greekuppercase=italicized]{mathdesign}
\usepackage{beramono}
\usepackage{berasans}

% Bold, sans-serif tensors. This fragment is taken from “egreg” from
% http://tex.stackexchange.com/a/82747/8945 and licensed under `CC-BY-SA
% <https://creativecommons.org/licenses/by-sa/3.0/>`_.
\usepackage{bm}
\DeclareMathAlphabet{\mathsfit}{\encodingdefault}{\sfdefault}{m}{sl}
\SetMathAlphabet{\mathsfit}{bold}{\encodingdefault}{\sfdefault}{bx}{sl}
\newcommand{\tens}[1]{\bm{\mathsfit{#1}}}

% Bold vectors.
\renewcommand{\vec}[1]{\boldsymbol{#1}}

%%%%%%%%%%%%%%%%%%%%%%%%%%%%%%%%%%%%%%%%%%%%%%%%%%%%%%%%%%%%%%%%%%%%%%%%%%%%%%%
%                               Input encoding                                %
%%%%%%%%%%%%%%%%%%%%%%%%%%%%%%%%%%%%%%%%%%%%%%%%%%%%%%%%%%%%%%%%%%%%%%%%%%%%%%%

\usepackage[T1]{fontenc}
\usepackage[utf8]{inputenc}

%%%%%%%%%%%%%%%%%%%%%%%%%%%%%%%%%%%%%%%%%%%%%%%%%%%%%%%%%%%%%%%%%%%%%%%%%%%%%%%
%                         Hyperrefs and PDF metadata                          %
%%%%%%%%%%%%%%%%%%%%%%%%%%%%%%%%%%%%%%%%%%%%%%%%%%%%%%%%%%%%%%%%%%%%%%%%%%%%%%%

\usepackage{hyperref}
\usepackage{lastpage}

% This sets the author in the properties of the PDF as well. If you want to
% change it, just override it with another ``\hypersetup`` call.
\hypersetup{
	breaklinks=false,
	citecolor=darkgreen,
	colorlinks=true,
	linkcolor=black,
	menucolor=black,
	pdfauthor={Martin Ueding},
	urlcolor=darkblue,
}

%%%%%%%%%%%%%%%%%%%%%%%%%%%%%%%%%%%%%%%%%%%%%%%%%%%%%%%%%%%%%%%%%%%%%%%%%%%%%%%
%                               Math Operators                                %
%%%%%%%%%%%%%%%%%%%%%%%%%%%%%%%%%%%%%%%%%%%%%%%%%%%%%%%%%%%%%%%%%%%%%%%%%%%%%%%

% AMS environments like ``align`` and theorems like ``proof``.
\usepackage{amsmath}
\usepackage{amsthm}

% Common math constructs like partial derivatives.
\usepackage{commath}

% Physical units.
\usepackage{siunitx}

% Word like operators.
\DeclareMathOperator{\acosh}{arcosh}
\DeclareMathOperator{\arcosh}{arcosh}
\DeclareMathOperator{\arcsinh}{arsinh}
\DeclareMathOperator{\arsinh}{arsinh}
\DeclareMathOperator{\asinh}{arsinh}
\DeclareMathOperator{\card}{card}
\DeclareMathOperator{\csch}{cshs}
\DeclareMathOperator{\diam}{diam}
\DeclareMathOperator{\sech}{sech}
\renewcommand{\Im}{\mathop{{}\mathrm{Im}}\nolimits}
\renewcommand{\Re}{\mathop{{}\mathrm{Re}}\nolimits}

% Fourier transform.
\DeclareMathOperator{\fourier}{\ensuremath{\mathcal{F}}}

% Roman versions of “e” and “i” to serve as Euler's number and the imaginary
% constant.
\newcommand{\ee}{\eup}
\newcommand{\eup}{\mathrm e}
\newcommand{\ii}{\iup}
\newcommand{\iup}{\mathrm i}

% Symbols for the various mathematical fields (natural numbers, integers,
% rational numbers, real numbers, complex numbers).
\newcommand{\C}{\ensuremath{\mathbb C}}
\newcommand{\N}{\ensuremath{\mathbb N}}
\newcommand{\Q}{\ensuremath{\mathbb Q}}
\newcommand{\R}{\ensuremath{\mathbb R}}
\newcommand{\Z}{\ensuremath{\mathbb Z}}

% Shape like operators.
\DeclareMathOperator{\dalambert}{\Box}
\DeclareMathOperator{\laplace}{\bigtriangleup}
\newcommand{\curl}{\vnabla \times}
\newcommand{\divergence}[1]{\inner{\vnabla}{#1}}
\newcommand{\vnabla}{\vec \nabla}

\newcommand{\half}{\frac 12}

% Unit vector (German „Einheitsvektor“).
\newcommand{\ev}{\hat{\vec e}}

% Scientific notation for large numbers.
\newcommand{\e}[1]{\cdot 10^{#1}}

% Mathematician's notation for the inner (scalar, dot) product.
\newcommand{\inner}[2]{\left\langle #1, #2 \right\rangle}

% Placeholders.
\newcommand{\emesswert}{\del{\messwert \pm \messwert}}
\newcommand{\fehlt}{\textcolor{darkred}{Hier fehlen noch Inhalte.}}
\newcommand{\messwert}{\textcolor{blue}{\square}}
\newcommand{\punkte}{\textcolor{white}{xxxxx}}

% Separator for equations on a single line.
\newcommand{\eqnsep}{,\quad}

% Quantum Mechanics
\newcommand{\bra}[1]{\left\langle #1 \right|}
\newcommand{\ket}[1]{\left| #1 \right\rangle}
\newcommand{\braket}[2]{\left\langle #1 \left. \vphantom{#1 #2} \right| #2 \right\rangle}

%%%%%%%%%%%%%%%%%%%%%%%%%%%%%%%%%%%%%%%%%%%%%%%%%%%%%%%%%%%%%%%%%%%%%%%%%%%%%%%
%                                  Headings                                   %
%%%%%%%%%%%%%%%%%%%%%%%%%%%%%%%%%%%%%%%%%%%%%%%%%%%%%%%%%%%%%%%%%%%%%%%%%%%%%%%

% This will set fancy headings to the top of the page. The page number will be
% accompanied by the total number of pages. That way, you will know if any page
% is missing.
%
% If you do not want this for your document, you can just use
% ``\pagestyle{plain}``.

\usepackage{scrpage2}

\pagestyle{scrheadings}

\automark{section}

\cfoot{\footnotesize{Seite \thepage\ / \pageref{LastPage}}}
\chead{}
\ihead{}
\ohead{\rightmark}

\setheadsepline{.4pt}

%%%%%%%%%%%%%%%%%%%%%%%%%%%%%%%%%%%%%%%%%%%%%%%%%%%%%%%%%%%%%%%%%%%%%%%%%%%%%%%
%                            Bibliography (BibTeX)                            %
%%%%%%%%%%%%%%%%%%%%%%%%%%%%%%%%%%%%%%%%%%%%%%%%%%%%%%%%%%%%%%%%%%%%%%%%%%%%%%%

\newcommand{\bibliographyfile}{../../zentrale_BibTeX/Central}

%%%%%%%%%%%%%%%%%%%%%%%%%%%%%%%%%%%%%%%%%%%%%%%%%%%%%%%%%%%%%%%%%%%%%%%%%%%%%%%
%                                Abbreviations                                %
%%%%%%%%%%%%%%%%%%%%%%%%%%%%%%%%%%%%%%%%%%%%%%%%%%%%%%%%%%%%%%%%%%%%%%%%%%%%%%%

\newcommand{\dhabk}{\mbox{d.\,h.}}


\usepackage{cancel}
\usepackage{tikz}

\newcommand{\themodul}{physik421}
\newcommand{\thegruppe}{Gruppe 4 -- Franz Niecknig}
\newcommand{\theuebung}{3}

\ifoot{\footnotesize{Martin Ueding}}
\ihead{\themodul{} -- Übung \theuebung}
\ofoot{\footnotesize{\thegruppe}}

\def\thesection{H \arabic{section}}
\def\thesubsection{\thesection\alph{subsection}}

\setcounter{section}{3}

\title{\themodul{} -- Übung \theuebung}
\subtitle{\thegruppe}
\author{
	Martin Ueding \footnote{\href{mailto:mu@uni-bonn.de}{mu@uni-bonn.de}}
}

\hypersetup{
	pdftitle={\themodul{} - Übung \theuebung},
}

\begin{document}

\maketitle

\begin{center}
	\ccbysadetitle
\end{center}

\begin{Form}
\begin{table}[h]
	\centering
	\begin{tabular}{l|c}
		Aufgabe
		& \ref 1 \\
		\hline
		Punkte
		& \TextField[name=aufgabe1, width=1cm]{} / 20
	\end{tabular}
\end{table}
\end{Form}

%%%%%%%%%%%%%%%%%%%%%%%%%%%%%%%%%%%%%%%%%%%%%%%%%%%%%%%%%%%%%%%%%%%%%%%%%%%%%%%
%                                Tunneleffekt                                 %
%%%%%%%%%%%%%%%%%%%%%%%%%%%%%%%%%%%%%%%%%%%%%%%%%%%%%%%%%%%%%%%%%%%%%%%%%%%%%%%

\section{Tunneleffekt}
\label 1

Zuerst vollziehe ich die Separation nach:
\begin{align*}
	\ii \hbar \dpd{}t \vartheta(t) \varphi_E(t) &= \del{-\frac{\hbar^2}{2m} \dpd[2]{}x + V(x)} \vartheta(t) \varphi_E(t) \\
	\ii \hbar \frac{\dot\vartheta(t)}{\vartheta(t)} &= -\frac{\hbar^2}{2m} \frac{\varphi_E''(x)}{\varphi_E(x)} + V(x) =: E
\end{align*}

Dies kann ich zerlegen in:
\[
	\vartheta(t) = c_1 \exp\del{-\ii\frac E\hbar t}
\]

Sowie außerhalb der Barriere:
\[
	\varphi_E(x) = c_2 \exp\del{\ii \frac{\sqrt{2Em}}\hbar x}
	+ c_3 \exp\del{-\ii \frac{\sqrt{2Em}}\hbar x}
\]

Und innerhalb:
\[
	\varphi_E(x) = c_4 \exp\del{\frac{\sqrt{2(V_0-E)m}}\hbar x}
	+ c_5 \exp\del{-\frac{\sqrt{2(V_0-E)m}}\hbar x}
\]

\subsection{Energien außerhalb des Intervalls}

\begin{tabular}{lll}
	Energie & außen & innen \\
	\hline
	negativ & Abfall & Abfall \\
	klein & Schwingung & Abfall \\
	groß & Schwingung & Schwingung
\end{tabular}

Dies entspricht meiner Vorstellung: Ein komplett freies Teichen (große Energie)
geht durch die Barriere durch. Ein Teilchen mit kleiner Energie ist außerhalb
der Barriere frei, kann allerdings nicht beliebig tief in die Barriere
eindringen.

\subsection{Lösungen}

\newcommand{\phiA}{\varphi_E^\text{(A)}}
\newcommand{\phiB}{\varphi_E^\text{(B)}}
\newcommand{\phiC}{\varphi_E^\text{(C)}}

\begin{description}
	\item[Bereich (A)]
		\[
			\phiA(x) = \alpha_+ \exp\del{\ii \frac{\sqrt{2Em}}\hbar x}
			+ \alpha_- \exp\del{-\ii \frac{\sqrt{2Em}}\hbar x}
		\]

	\item[Bereich (B)]
		\[
			\phiB(x) = \beta_+ \exp\del{\frac{\sqrt{2(V_0-E)m}}\hbar x}
			+ \beta_- \exp\del{-\frac{\sqrt{2(V_0-E)m}}\hbar x}
		\]

	\item[Bereich (C)]
		\[
			\phiC(x) = \gamma_+ \exp\del{\ii \frac{\sqrt{2Em}}\hbar x}
			+ \gamma_- \exp\del{-\ii \frac{\sqrt{2Em}}\hbar x}
		\]
\end{description}

Die Wellenzahlen $k$ sind letztlich der Vorfaktor im Argument der
Exponentialfunktion ohne $\ii$:
\[
	k(E) = \frac{\sqrt{2Em}}\hbar
	\eqnsep
	k'(E) = - \ii \frac{\sqrt{2(V_0-E)m}}\hbar
\]

\subsection{Anschlussbeziehungen}

Für die Anschlussbeziehungen herzuleiten, schreibe ich die Funktionen
$\varphi_E$ mit den $k$ kompakter:

\begin{description}
	\item[Bereich (A)]
		\[
			\phiA(x) = \alpha_+ \exp\del{\ii k x}
			+ \alpha_- \exp\del{-\ii k x}
		\]

	\item[Bereich (B)]
		\[
			\phiB(x) = \beta_+ \exp\del{\ii k' x}
			+ \beta_- \exp\del{-\ii k' x}
		\]

	\item[Bereich (C)]
		\[
			\phiC(x) = \gamma_+ \exp\del{\ii k x}
			+ \gamma_- \exp\del{-\ii k x}
		\]
\end{description}

An den Übergangsstellen $\pm L/2$ muss die Wellenfunktion $\varphi_E$ stetig
sein und gleiche Steigung haben.  Es muss also gelten:
\begin{align*}
	\phiA\del{-\frac L2} &= \phiB\del{-\frac L2} \\
	\phiB\del{\frac L2} &= \phiC\del{\frac L2} \\
	\dod{}x \phiA\del{-\frac L2} &= \dod{}x \phiB\del{-\frac L2} \\
	\dod{}x \phiB\del{\frac L2} &= \dod{}x \phiC\del{\frac L2} \\
	\intertext{%
		Setze die Funktionen $\phiA$ und $\phiB$ ein:
	}
	\alpha_+ \exp\del{- \ii k \frac L2} + \alpha_- \exp\del{\ii k \frac L2}
	&= \beta_+ \exp\del{- \ii k' \frac L2} + \beta_- \exp\del{\ii k' \frac L2} \\
	%
	\beta_+ \exp\del{\ii k' \frac L2} + \beta_- \exp\del{- \ii k' \frac L2}
	&= \gamma_+ \exp\del{\ii k \frac L2} + \gamma_- \exp\del{- \ii k \frac L2} \\
	%
	-\ii k \alpha_+ \exp\del{- \ii k \frac L2} + \ii k \alpha_- \exp\del{\ii k \frac L2}
	&= -\ii k' \beta_+ \exp\del{- \ii k' \frac L2} + \ii k' \beta_- \exp\del{\ii k' \frac L2} \\
	%
	\ii k' \beta_+ \exp\del{\ii k' \frac L2} - \ii k' \beta_- \exp\del{- \ii k' \frac L2}
	&= \ii k \gamma_+ \exp\del{\ii k \frac L2} - \ii k \gamma_- \exp\del{- \ii k \frac L2} \\
	\intertext{%
		Bei der weiteren Herleitung habe ich mich stark an \cite[Seite
		269]{nolting-theo5} gehalten. Dann teile ich die dritte und die vierte
		Gleichung jeweils durch $- \ii k$ und führe $y = k'/k$ ein.
	}
	\alpha_+ \exp\del{- \ii k \frac L2} + \alpha_- \exp\del{\ii k \frac L2}
	&= \beta_+ \exp\del{- \ii k' \frac L2} + \beta_- \exp\del{\ii k' \frac L2} \\
	%
	\beta_+ \exp\del{\ii k' \frac L2} + \beta_- \exp\del{- \ii k' \frac L2}
	&= \gamma_+ \exp\del{\ii k \frac L2} + \gamma_- \exp\del{- \ii k \frac L2} \\
	%
	\alpha_+ \exp\del{- \ii k \frac L2} - \alpha_- \exp\del{\ii k \frac L2}
	&= y \beta_+ \exp\del{- \ii k' \frac L2} -y \beta_- \exp\del{\ii k' \frac L2} \\
	%
	- y\beta_+ \exp\del{\ii k' \frac L2} + y\beta_- \exp\del{- \ii k' \frac L2}
	&= - \gamma_+ \exp\del{\ii k \frac L2} +  \gamma_- \exp\del{- \ii k \frac L2} \\
	\intertext{%
		Ab hier führe ich folgende Abkürzungen ein:
		\[
			\tau_+ := \exp\del{\ii k \frac L2}
			\eqnsep
			\tau_- := \exp\del{-\ii k \frac L2}
			\eqnsep
			\tau_+' := \exp\del{\ii k' \frac L2}
			\eqnsep
			\tau_-' := \exp\del{-\ii k' \frac L2}
		\]
		Damit werden die vier Gleichungen zu:
	}
	\alpha_+ \tau_- + \alpha_- \tau_+
	&= \beta_+ \tau_-' + \beta_- \tau_+' \\
	%
	\beta_+ \tau_+' + \beta_- \tau_-'
	&= \gamma_+ \tau_+ + \gamma_- \tau_- \\
	%
	\alpha_+ \tau_- - \alpha_- \tau_+
	&= y \beta_+ \tau_-' -y \beta_- \tau_+' \\
	%
	- y\beta_+ \tau_+' + y\beta_- \tau_-'
	&= - \gamma_+ \tau_+ +  \gamma_- \tau_- \\
	\intertext{%
		In dieser Aufgabe sollte dieser Schritt zwar noch nicht vollzogen
		werden, damit ich dieses Gleichungssystem allerdings etwas einfacher
		lösen kann, mache ich bereits hier die Annahme, dass die Welle von
		rechts einläuft, also $\gamma_- = 0$. Außerdem lege ich $\alpha_+ = 1$
		fest.
	}
	\tau_- + \alpha_- \tau_+
	&= \beta_+ \tau_-' + \beta_- \tau_+' \\
	%
	\beta_+ \tau_+' + \beta_- \tau_-'
	&= \gamma_+ \tau_+ \\
	%
	\tau_- - \alpha_- \tau_+
	&= y \beta_+ \tau_-' -y \beta_- \tau_+' \\
	%
	- y\beta_+ \tau_+' + y\beta_- \tau_-'
	&= - \gamma_+ \tau_+ \\
	\intertext{%
		Betrachte nun Erste plus Dritte, sowie Zweite plus Vierte.
	}
	2 \tau_- & = (1+y) \beta_+ \tau_-' + (1-y) \beta_- \tau_+' \\
	(1-y) \beta_+ \tau_+' (1+y) \beta_- \tau_-' & = 0
\end{align*}

Dieses Gleichungssystem in $\beta_+$ und $\beta_-$ schreibe ich in Matrixform:
\[
	\underbrace{
		\begin{pmatrix}
			(1+y) \tau_-' & (1-y) \tau_+' \\
			(1-y) \tau_+' & (1+y) \tau_-' 		
		\end{pmatrix}
	}_{\tens A}
	\begin{pmatrix}
		\beta_+ \\ \beta_-
	\end{pmatrix}
	=
	\begin{pmatrix}
		2 \tau_-  \\ 0
	\end{pmatrix}
\]

\newcommand\detA{\det{\tens A}}

Dieses System löse ich mit der Cramer'schen Regel:
\[
	\detA = 2(1+y)^2 \tau_-^{\prime2} - (1-y)^2 \tau_+^{\prime 2}
\]

Die Lösungen sind:
\begin{align*}
	\beta_+ &= \frac1\detA 2 \tau_- \tau_-' (1+y) \\
	\beta_- &= - \frac1\detA 2 \tau_- \tau_+' (1-y) \\
\end{align*}

Diese Ergebnisse bringe ich jetzt in folgendes System ein:
\begin{align*}
	\beta_+ \tau_+' + \beta_- \tau_-'
	&= \gamma_+ \tau_+ + \gamma_- \tau_- \\
	y\beta_+ \tau_+' - y\beta_- \tau_-'
	&= \gamma_+ \tau_+ - \gamma_- \tau_-
\end{align*}

Nach Umformen erhalte ich:
\[
	\gamma_+ = \frac{4 \tau_-^2 y}\detA
\]

Laut \cite[Seite 270]{nolting-theo5} soll für $\alpha_-$ folgendes
herauskommen, was ich selbst allerdings nicht erhalten konnte:
\[
	\alpha_- = - 2 \ii \sin\del{2 k' \frac L2} \frac{1-y^2}\detA \tau_-^2
\]

\[
	k(E) = \frac{\sqrt{2Em}}\hbar
	\eqnsep
	k'(E) = - \ii \frac{\sqrt{2(V_0-E)m}}\hbar
\]

\subsection{Von links einlaufendes Teilchen}

Die Vereinfachung ist $\gamma_- = 0$. Diese hatte ich zur Reduktion des
Rechenaufwands allerdings schon in der vorherigen Teilaufgabe vorgenommen, so
dass ich hier nicht vergleichen kann.

\subsection{Transmissions- und Reflektionskoeffizient}

\begin{align*}
	T
	&= \abs{\gamma_+}^2 \\
	&= \abs{\frac{4y}\detA}^2 \\
	&= 16 \abs{\frac{y}\detA}^2 \\
\end{align*}

\subsection{Grenzfall der Koeffizienten}

%%%%%%%%%%%%%%%%%%%%%%%%%%%%%%%%%%%%%%%%%%%%%%%%%%%%%%%%%%%%%%%%%%%%%%%%%%%%%%%
%                                    Ende                                     %
%%%%%%%%%%%%%%%%%%%%%%%%%%%%%%%%%%%%%%%%%%%%%%%%%%%%%%%%%%%%%%%%%%%%%%%%%%%%%%%

\IfFileExists{\bibliographyfile}{
	\bibliography{\bibliographyfile}
}{}

\end{document}

% vim: spell spelllang=de
