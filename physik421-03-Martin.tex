% Copyright © 2012-2013 Martin Ueding <dev@martin-ueding.de>
%
\input{header.tex}

\usepackage{cancel}
\usepackage{tikz}

\newcommand{\themodul}{physik421}
\newcommand{\thegruppe}{Gruppe 4 -- Franz Niecknig}
\newcommand{\theuebung}{3}

\ifoot{\footnotesize{Martin Ueding}}
\ihead{\themodul{} -- Übung \theuebung}
\ofoot{\footnotesize{\thegruppe}}

\def\thesection{H \arabic{section}}
\def\thesubsection{\thesection\alph{subsection}}

\setcounter{section}{3}

\title{\themodul{} -- Übung \theuebung}
\subtitle{\thegruppe}
\author{
	Martin Ueding \footnote{\href{mailto:mu@uni-bonn.de}{mu@uni-bonn.de}}
}

\hypersetup{
	pdftitle={\themodul{} - Übung \theuebung},
}

\begin{document}

\maketitle

\begin{center}
	\ccbysadetitle
\end{center}

\begin{Form}
\begin{table}[h]
	\centering
	\begin{tabular}{l|c}
		Aufgabe
		& \ref 1 \\
		\hline
		Punkte
		& \TextField[name=aufgabe1]{} / 20
	\end{tabular}
\end{table}
\end{Form}

%%%%%%%%%%%%%%%%%%%%%%%%%%%%%%%%%%%%%%%%%%%%%%%%%%%%%%%%%%%%%%%%%%%%%%%%%%%%%%%
%                                Tunneleffekt                                 %
%%%%%%%%%%%%%%%%%%%%%%%%%%%%%%%%%%%%%%%%%%%%%%%%%%%%%%%%%%%%%%%%%%%%%%%%%%%%%%%

\section{Tunneleffekt}
\label 1

Zuerst vollziehe ich die Separation nach:
\begin{align*}
	\ii \hbar \dpd{}t \vartheta(t) \varphi_E(t) &= \del{-\frac{\hbar^2}{2m} \dpd[2]{}x + V(x)} \vartheta(t) \varphi_E(t) \\
	\ii \hbar \frac{\dot\vartheta(t)}{\vartheta(t)} &= -\frac{\hbar^2}{2m} \frac{\varphi_E''(x)}{\varphi_E(x)} + V(x) =: E
\end{align*}

Dies kann ich zerlegen in:
\[
	\vartheta(t) = c_1 \exp\del{-\ii\frac E\hbar t}
\]

Sowie außerhalb der Barriere:
\[
	\varphi_E(x) = c_2 \exp\del{\ii \frac{\sqrt{2Em}}\hbar x}
	+ c_3 \exp\del{-\ii \frac{\sqrt{2Em}}\hbar x}
\]

Und innerhalb:
\[
	\varphi_E(x) = c_4 \exp\del{\frac{\sqrt{2(V_0-E)m}}\hbar x}
	+ c_5 \exp\del{-\frac{\sqrt{2(V_0-E)m}}\hbar x}
\]

\subsection{Energien außerhalb des Intervalls}

\begin{tabular}{lll}
	Energie & außen & innen \\
	\hline
	negativ & Abfall & Abfall \\
	klein & Schwingung & Abfall \\
	groß & Schwingung & Schwingung
\end{tabular}

Dies entspricht meiner Vorstellung: Ein komplett freies Teichen (große Energie)
geht durch die Barriere durch. Ein Teilchen mit kleiner Energie ist außerhalb
der Barriere frei, kann allerdings nicht beliebig tief in die Barriere
eindringen.

\subsection{Lösungen}

\newcommand{\phiA}{\varphi_E^\text{(A)}}
\newcommand{\phiB}{\varphi_E^\text{(B)}}
\newcommand{\phiC}{\varphi_E^\text{(C)}}

\begin{description}
	\item[Bereich (A)]
		\[
			\phiA(x) = \alpha_+ \exp\del{\ii \frac{\sqrt{2Em}}\hbar x}
			+ \alpha_- \exp\del{-\ii \frac{\sqrt{2Em}}\hbar x}
		\]

	\item[Bereich (B)]
		\[
			\phiB(x) = \beta_+ \exp\del{\frac{\sqrt{2(V_0-E)m}}\hbar x}
			+ \beta_- \exp\del{-\frac{\sqrt{2(V_0-E)m}}\hbar x}
		\]

	\item[Bereich (C)]
		\[
			\phiC(x) = \gamma_+ \exp\del{\ii \frac{\sqrt{2Em}}\hbar x}
			+ \gamma_- \exp\del{-\ii \frac{\sqrt{2Em}}\hbar x}
		\]
\end{description}

Die Wellenzahlen $k$ sind letztlich der Vorfaktor im Argument der
Exponentialfunktion ohne $\ii$:
\[
	k_\text{außen}(E) = \pm \frac{\sqrt{2Em}}\hbar
	\eqnsep
	k_\text{innen}(E) = \pm \ii \frac{\sqrt{2(V_0-E)m}}\hbar
\]

Diese Größen kann ich noch nach $E$ ableiten und erhalte:
\[
	k'_\text{außen}(E) = \pm \frac{m}{\hbar\sqrt{2Em}}
	\eqnsep
	k'_\text{innen}(E) = \mp \ii \frac{m}{\hbar\sqrt{2(V_0-E)m}}
\]

\subsection{Anschlussbeziehungen}

Für die Anschlussbeziehungen herzuleiten, schreibe ich die Funktionen
$\varphi_E$ mit den $k$ kompakter:

\begin{description}
	\item[Bereich (A)]
		\[
			\phiA(x) = \alpha_+ \exp\del{\ii k_\text{a} x}
			+ \alpha_- \exp\del{-\ii k_\text{a} x}
		\]

	\item[Bereich (B)]
		\[
			\phiB(x) = \beta_+ \exp\del{\ii k_\text{i} x}
			+ \beta_- \exp\del{-\ii k_\text{i} x}
		\]

	\item[Bereich (C)]
		\[
			\phiC(x) = \gamma_+ \exp\del{\ii k_\text{a} x}
			+ \gamma_- \exp\del{-\ii k_\text{a} x}
		\]
\end{description}

An den Übergangsstellen $\pm L/2$ muss die Wellenfunktion $\varphi_E$ stetig
sein und gleiche Steigung haben. Es muss also gelten:
\begin{gather*}
	\phiA\del{-\frac L2} = \phiB\del{-\frac L2} \\
	\phiB\del{\frac L2} = \phiC\del{\frac L2} \\
	\dod{}x \phiA\del{-\frac L2} = \dod{}x \phiB\del{-\frac L2} \\
	\dod{}x \phiB\del{\frac L2} = \dod{}x \phiC\del{\frac L2} \\
\end{gather*}

%%%%%%%%%%%%%%%%%%%%%%%%%%%%%%%%%%%%%%%%%%%%%%%%%%%%%%%%%%%%%%%%%%%%%%%%%%%%%%%
%                                    Ende                                     %
%%%%%%%%%%%%%%%%%%%%%%%%%%%%%%%%%%%%%%%%%%%%%%%%%%%%%%%%%%%%%%%%%%%%%%%%%%%%%%%

\IfFileExists{\bibliographyfile}{
	%\bibliography{\bibliographyfile}
}{}

\end{document}

% vim: spell spelllang=de
