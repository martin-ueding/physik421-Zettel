 % Copyright © 2012-2013 Martin Ueding <dev@martin-ueding.de>
%
\input{header.tex}

\usepackage{cancel}
\usepackage{tikz}

\newcommand{\themodul}{physik421}
\newcommand{\thegruppe}{Gruppe 4 -- Franz Niecknig}
\newcommand{\theuebung}{9}

\ifoot{\footnotesize{Martin Ueding}}
\ihead{\themodul{} -- Übung \theuebung}
\ofoot{\footnotesize{\thegruppe}}

\def\thesection{H \arabic{section}}
\def\thesubsection{\thesection\alph{subsection}}

\setcounter{section}{13}

\title{\themodul{} -- Übung \theuebung}
\subtitle{\thegruppe}
\author{
	Martin Ueding \footnote{\href{mailto:mu@uni-bonn.de}{mu@uni-bonn.de}}
}

\hypersetup{
	pdftitle={\themodul{} - Übung \theuebung},
}

\setcounter{secnumdepth}{2}

\begin{document}

\maketitle

\begin{center}
	\ccbysadetitle
\end{center}

%%%%%%%%%%%%%%%%%%%%%%%%%%%%%%%%%%%%%%%%%%%%%%%%%%%%%%%%%%%%%%%%%%%%%%%%%%%%%%%
%              Nichtentartete zeitunabhängige Störungstheorie               %
%%%%%%%%%%%%%%%%%%%%%%%%%%%%%%%%%%%%%%%%%%%%%%%%%%%%%%%%%%%%%%%%%%%%%%%%%%%%%%%

\section{Nichtentartete zeitunabhängige Störungstheorie}
\label 1

\subsection{Energiespektrum}

\subsection{Korrektur exakt null}

\subsection{Explizite Energiekorrektur}

%%%%%%%%%%%%%%%%%%%%%%%%%%%%%%%%%%%%%%%%%%%%%%%%%%%%%%%%%%%%%%%%%%%%%%%%%%%%%%%
%                       Kopplung zweier Spin-1 Teilchen                       %
%%%%%%%%%%%%%%%%%%%%%%%%%%%%%%%%%%%%%%%%%%%%%%%%%%%%%%%%%%%%%%%%%%%%%%%%%%%%%%%

\section{Kopplung zweier Spin-1 Teilchen}
\label 2

\subsection{Kombinationen, Multipletts}

Zwei Spin-1 Teilchen haben $j_1 = j_2 = 1$. Für den Gesamtdrehimpuls $j$ gilt dann:
\[
	0 \leq j \leq 2
\]

Die Multipletts sind:
\[
	\begin{cases}
		j = 0 & \text{Singulett} \\
		j = 1 & \text{Triplett} \\
		j = 2 & \text{Pentlett}
	\end{cases}
\]

Da $m_1$ und $m_2$ aus $\set{-1, 0, 1}$ gewählt werden können, gibt es folgende Zustände:
\begin{gather*}
	\ket{0, 0} \\
	\ket{1, 1} 
	\eqnsep
	\ket{1, 0} 
	\eqnsep
	\ket{1, -1} \\
	\ket{2, 2} 
	\eqnsep
	\ket{2, 1} 
	\eqnsep
	\ket{2, 0} 
	\eqnsep
	\ket{2, -2}
	\eqnsep
	\ket{2, -1} \\
\end{gather*}

\subsection{Zustandsbildung}

Der Zustand $\ket{2, 2}$ ist aus dem Pentlett und kann nur mit $m_1 = m_2 = 1$ gebildet werden. Also gilt:
\[
	\ket{2, 2} = \ket{1, 1} \ket{1, 1}
\]

\subsection{Absteigeoperator}

Es ist zu zeigen, dass der Absteigeoperator einfach die Summe der einzelnen
ist:
\begin{align*}
	J_-^{(1)} + J_-^{(2)}
	&= J_1^{(1)} - \ii J_2^{(1)} + J_1^{(2)} - \ii J_2^{(2)} \\
	&= J_1^{(1)} + J_1^{(2)} - \ii J_2^{(1)} - \ii J_2^{(2)} \\
	&= J_1 - \ii J_2 \\
	&= J_-
\end{align*}

Der Absteigeoperator erzeugt ein $\hbar \sqrt{j(j+1) - m(m-1)}$. Ich wende ihn auf den Zustand $\ket{2, 2}$ an:
\begin{align*}
	J_- \ket{2, 2}
	&= \del{J_-^{(1)} + J_-^{(2)}} \ket{1, 1} \ket{1, 1} \\
	\hbar 2 \ket{2, 1}
	&= \hbar \sqrt 2 \ket{1, 0} \ket{1, 1} + \hbar \sqrt 2 \ket{1, 1} \ket{1, 0} \\
	\ket{2, 1}
	&= \frac1{\sqrt 2} \ket{1, 0} \ket{1, 1} + \frac1{\sqrt 2} \ket{1, 1} \ket{1, 0} \\
	\intertext{%
		Wende den Absteigeoperator erneut an und teile durch den Vorfaktor auf
		der linken Seite. Zuerst entstehen vier Summanden auf der rechten
		Seite, diese fasse ich zusammen.
	}
	\ket{2, 0}
	&= \frac1{\sqrt 6} \ket{1, -1} \ket{1, 1} + \frac2{\sqrt 6} \ket{1, 0} \ket{1, 0} + \frac1{\sqrt 2} \ket{1, 1} \ket{1, -1} \\
	\intertext{%
		Erneute Anwendung des Absteigeoperators.
	}
	\ket{2, -1}
	&= \frac1{\sqrt 2} \ket{1, -1} \ket{1, 0} + \frac1{\sqrt 2} \ket{1, 0} \ket{1, -1} \\
	\ket{2, -2}
	&= \ket{1, -1} \ket{1, -1}
\end{align*}

\subsection{Orthonormalität}

Gegeben ist:
\[
	\ket{1, 1} = a \ket{1, 1} \ket{1, 0} + b \ket{1, 0} \ket{1, 1}
\]

Wir wissen, dass $\ket{j, m}$ orthogonal sind. Der Zustand $\ket{2, 1}$ besteht
aus den gleichen Zuständen, also muss der gegebene Zustand auf jeden Fall zu
diesem Orthogonal sein:
\begin{align*}
	\braket{2, 1}{1, 1} &= 0 \\
	\del{a \ket{1, 1} \ket{1, 0} + b \ket{1, 0} \ket{1, 1}}^\dagger \del{\frac1{\sqrt 2} \ket{1, 0} \ket{1, 1} + \frac1{\sqrt 2} \ket{1, 1} \ket{1, 0}} &= 0 \\
\end{align*}

\subsection{Absteigeoperator}

\subsection{Orthonormalität}

%%%%%%%%%%%%%%%%%%%%%%%%%%%%%%%%%%%%%%%%%%%%%%%%%%%%%%%%%%%%%%%%%%%%%%%%%%%%%%%
%                                    Ende                                     %
%%%%%%%%%%%%%%%%%%%%%%%%%%%%%%%%%%%%%%%%%%%%%%%%%%%%%%%%%%%%%%%%%%%%%%%%%%%%%%%

\IfFileExists{\bibliographyfile}{
	%\bibliography{\bibliographyfile}
}{}

\end{document}

% vim: spell spelllang=de
