% Copyright © 2012-2013 Martin Ueding <dev@martin-ueding.de>
%
\input{header.tex}

\usepackage{tikz}

\newcommand{\themodul}{physik421}
\newcommand{\thegruppe}{Gruppe 4 -- Franz Niecknig}
\newcommand{\theuebung}{1}

\ifoot{\footnotesize{Martin Ueding}}
\ihead{\themodul{} -- Übung \theuebung}
\ofoot{\footnotesize{\thegruppe}}

\def\thesection{H \arabic{section}}
\def\thesubsection{\thesection\alph{subsection}}

\title{\themodul{} -- Übung \theuebung \\ \vspace{0.5cm} \large{\thegruppe}}

\author{
	Martin Ueding \\ \small{\href{mailto:mu@uni-bonn.de}{mu@uni-bonn.de}}
}

\hypersetup{
	pdftitle={\themodul {} - Übung \theuebung},
}

\begin{document}

\maketitle

\begin{table}[h]
	\centering
	\begin{tabular}{l|c|c}
		Aufgabe
		& \ref 1
		& $\sum$   \\
		\hline
		Punkte
		& \punkte / 20
		& \punkte / 20
	\end{tabular}
\end{table}

\section{Dirac'sche Deltadistribution}
\label 1

\subsection{Stufenfunktion}

Es soll gezeigt werden, dass die Ableitung von $G_a$ der Deltadistribution
$\delta_a$ entspricht.

Wir beginnen mit der Definition von $G_a$:
\begin{align*}
	G_a[\phi]
	&= \int_{-\infty}^\infty \dif x \, \theta(x - a) \phi(x) \\
	G_a'[\phi]
	&= \dod{}a \int_{-\infty}^\infty \dif x \, \theta(x - a) \phi(x) \\
	&= - \int_{-\infty}^\infty \dif x \, \theta'(x - a) \phi(x) \\
	&= \int_{-\infty}^\infty \dif x \,
	\lim_{h \to 0} \frac{\theta(x + h - a) - \theta(x - a)}h \phi(x) \\
	\intertext{%
		Diese Ableitung ist für $x \neq a$ gleich 0. Wenn $x \in \intcc{a-h/2,
		a+h/2}$ liegt, dann ist der Bruch gerade $1/h$. Das Integral geht dann
		allerdings nur über einen Bereich, der $h$ breit ist. $\phi$ war als
		Schwartzfunktion, also aus $C^\infty$ gegeben. Daher kann sie in einem
		verschwindend kleinen Intervall durch ihre nullte Näherung an der
		Stelle $a$ ersetzt werden.
	}
	&= - h \frac 1h \phi(a) \\
	&= - \phi(a)
\end{align*}

Und dies ist gerade die Definition der Deltadistribution.

\subsection{Fouriertransformatierte von $g_\epsilon$}

Es soll gezeigt werden, dass die Fouriertransformierte von $g_\epsilon(x)$ im
Limes $\epsilon \to 0$ der Deltadistribution entspricht.

Dazu bilde ich die zuerst Fouriertransformierte von $g_\epsilon$.
\begin{align*}
	\fourier\sbr{g_\epsilon}(\omega)
	&= \frac{1}{\sqrt{2\piup}} \int_{-\infty}^\infty \dif x \, \exp\del{-\half \epsilon^2 x^2 - \ii \omega x} \\
	&= \frac{1}{\sqrt{2\piup}} \int_{-\infty}^\infty \dif x \, \exp\del{-\del{\half \epsilon^2 x^2 + \ii \omega x}} \\
	\intertext{%
		Ich vervollständige das Quadrat in der Exponentialfunktion.
	}
	&= \frac{1}{\sqrt{2\piup}} \int_{-\infty}^\infty \dif x \, \exp\del{-\del{\frac1{\sqrt{2}} \epsilon x + \ii \frac\omega\epsilon}^2 - \frac{\omega^2}{\epsilon^2}} \\
	&= \frac{1}{\sqrt{2\piup}} \exp\del{-\frac{\omega^2}{\epsilon^2}} \int_{-\infty}^\infty \dif x \, \exp\del{-\del{\frac1{\sqrt{2}} \epsilon x + \ii \frac\omega\epsilon}^2} \\
	\intertext{%
		An dieser Stelle substituiere ich $z := \epsilon x/ \sqrt{2} + \ii
		\omega/\epsilon$. Das Integral erhält dann den Wert $\sqrt{\piup}$.
		Durch $\dif z / \dif x$ erhalte ich noch einen Faktor
		$\sqrt{2}\epsilon$.
	}
	&= \frac{1}{\sqrt{\piup} \epsilon} \exp\del{-\frac{\omega^2}{\epsilon^2}} \\
\end{align*}

Die Funktionenschar habe ich in Abbildung \ref{fig:Fourier} geplottet.

\begin{figure}
	\centering
	\includegraphics[width=0.8\textwidth]{Fouriertransformierte.pdf}
	\caption{$\fourier\sbr{g_\epsilon}(\omega)$ für $\epsilon = \num{0.1}, \num{0.2}, \ldots, \num{1.1}$}
	\label{fig:Fourier}
\end{figure}

Zur Probe habe ich das Integral über diese Funktion mit Mathematica rechnen
lassen, es kommt unabhängig von $\epsilon$ 1 heraus. Damit ist die eine
Bedingung erfüllt, im Grenzwert sind auch alle Funktionswerte für $x \neq 0$
gerade 0.

\subsection{Beweis}

\begin{problem}
	Es soll folgende Identität gezeigt werden:
	\[
		\frac1{2\piup} \lim_{\epsilon\to0} \fourier\sbr{g_\epsilon \fourier\sbr\psi}(-k)
		= \psi(k)
	\]
\end{problem}

\fehlt

\subsection{Inverse Fouriertransformation}

Es soll die inverse Fouriertransformation hergeleitet werden.

Die inverse Fouriertransformation ist wie die normale Fouriertransformation und
eine Parität $\mathcal P$, also $\fourier^{-1} = \fourier \circ
\mathop{\mathcal P}$. Wenn wir dies annehmen, können wir vereinfachen:
\begin{align*}
	\frac1{2\piup} \lim_{\epsilon\to0} \fourier\sbr{g_\epsilon \fourier\sbr\psi}(-k)
	&= \psi(k) \\
	\frac1{2\piup} \lim_{\epsilon\to0} \fourier\sbr{g_\epsilon \fourier\sbr\psi} \circ \mathcal P
	&= \psi \\
	\lim_{\epsilon\to0} \fourier^{-1}\sbr{g_\epsilon \fourier\sbr\psi} \circ \mathcal P
	&= \psi \\
	\fourier^{-1}\sbr{\lim_{\epsilon\to0} g_\epsilon \fourier\sbr\psi}
	&= \psi \\
	\fourier^{-1}\sbr{\fourier\sbr\psi}
	&= \psi \\
	\psi &= \psi
\end{align*}

\subsection{Distribution $D_k$}

Es soll gezeigt werden, dass $\fourier\sbr{D_k} = 2 \piup \delta(x-k)$.

Gegeben ist die Distribution $D_k$ mit:
\begin{align*}
	D_k[\phi](k) &= \int_{\infty}^\infty \dif x \, \exp\del{\ii kx} \phi(x)
	\intertext{%
		Auf diese Distribution wenden wir nun die Fouriertransformation an:
	}
	\fourier\sbr{D_k[\phi]}(\omega)
	&= \frac1{\sqrt{2\piup}} \int_{\infty}^\infty \dif k \, \exp\del{-\ii k \omega} \int_{\infty}^\infty \dif x \, \exp\del{\ii kx} \phi(x) \\
	&= \frac1{\sqrt{2\piup}} \int_{\infty}^\infty \dif k \int_{\infty}^\infty \dif x \, \exp\del{\ii k(x-\omega)} \phi(x) \\
	&= \frac1{\sqrt{2\piup}} \int_{\infty}^\infty \dif x \, \phi(x) \int_{\infty}^\infty \dif k \, \exp\del{\ii k(x-\omega)} \\
	\intertext{%
		Das hintere Integral über $k$ nimmt für den Fall $x-\omega=0$ den Wert
		$\infty$ an. Für alle anderen Werte von $x-\omega$ ist das Integral
		undefiniert oder 0. Dies entspricht $\delta(x-\omega)$.
	}
	&= \frac1{\sqrt{2\piup}} \int_{\infty}^\infty \dif x \, \phi(x) \delta(x-\omega) \\
	&= \frac1{\sqrt{2\piup}} \delta_\omega[\phi]
\end{align*}

Der Faktor $(2\piup)^{-3/2}$, der zu viel ist, ist eventuell durch eine leicht
andere Definition der Fouriertransformation entstanden.

%%%%%%%%%%%%%%%%%%%%%%%%%%%%%%%%%%%%%%%%%%%%%%%%%%%%%%%%%%%%%%%%%%%%%%%%%%%%%%%
%                                    Ende                                     %
%%%%%%%%%%%%%%%%%%%%%%%%%%%%%%%%%%%%%%%%%%%%%%%%%%%%%%%%%%%%%%%%%%%%%%%%%%%%%%%

%\IfFileExists{\bibliographyfile}{
	%\bibliography{\bibliographyfile}
	%\bibliographystyle{plain}
%}{}

\end{document}

% vim: spell spelllang=de
