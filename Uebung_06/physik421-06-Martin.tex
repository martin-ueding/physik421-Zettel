 % Copyright © 2012-2013 Martin Ueding <dev@martin-ueding.de>
%
% Copyright © 2012-2013 Martin Ueding <dev@martin-ueding.de>

% This is my general purpose LaTeX header file for writing German documents.
% Ideally, you include this using a simple ``% Copyright © 2012-2013 Martin Ueding <dev@martin-ueding.de>

% This is my general purpose LaTeX header file for writing German documents.
% Ideally, you include this using a simple ``% Copyright © 2012-2013 Martin Ueding <dev@martin-ueding.de>

% This is my general purpose LaTeX header file for writing German documents.
% Ideally, you include this using a simple ``\input{header.tex}`` in your main
% document and start with ``\title`` and ``\begin{document}`` afterwards.

% If you need to add additional packages, I recommend not doing this in this
% file, but in your main document. That way, you can just drop in a new
% ``header.tex`` and get all the new commands without having to merge manually.

% Since this file encorporates a CC-BY-SA fragment, this whole files is
% licensed under the CC-BY-SA license.

\documentclass[11pt, ngerman, fleqn, DIV=15, headinclude]{scrartcl}

\usepackage{graphicx}

%%%%%%%%%%%%%%%%%%%%%%%%%%%%%%%%%%%%%%%%%%%%%%%%%%%%%%%%%%%%%%%%%%%%%%%%%%%%%%%
%                                Locale, date                                 %
%%%%%%%%%%%%%%%%%%%%%%%%%%%%%%%%%%%%%%%%%%%%%%%%%%%%%%%%%%%%%%%%%%%%%%%%%%%%%%%

\usepackage{babel}
\usepackage[iso]{isodate}

%%%%%%%%%%%%%%%%%%%%%%%%%%%%%%%%%%%%%%%%%%%%%%%%%%%%%%%%%%%%%%%%%%%%%%%%%%%%%%%
%                          Margins and other spacing                          %
%%%%%%%%%%%%%%%%%%%%%%%%%%%%%%%%%%%%%%%%%%%%%%%%%%%%%%%%%%%%%%%%%%%%%%%%%%%%%%%

\usepackage[activate]{pdfcprot}
\usepackage[parfill]{parskip}
\usepackage{setspace}

\setlength{\columnsep}{2cm}

%%%%%%%%%%%%%%%%%%%%%%%%%%%%%%%%%%%%%%%%%%%%%%%%%%%%%%%%%%%%%%%%%%%%%%%%%%%%%%%
%                                    Color                                    %
%%%%%%%%%%%%%%%%%%%%%%%%%%%%%%%%%%%%%%%%%%%%%%%%%%%%%%%%%%%%%%%%%%%%%%%%%%%%%%%

\usepackage{color}

\definecolor{darkblue}{rgb}{0,0,.5}
\definecolor{darkgreen}{rgb}{0,.5,0}
\definecolor{darkred}{rgb}{.7,0,0}

%%%%%%%%%%%%%%%%%%%%%%%%%%%%%%%%%%%%%%%%%%%%%%%%%%%%%%%%%%%%%%%%%%%%%%%%%%%%%%%
%                         Font and font like settings                         %
%%%%%%%%%%%%%%%%%%%%%%%%%%%%%%%%%%%%%%%%%%%%%%%%%%%%%%%%%%%%%%%%%%%%%%%%%%%%%%%

% This replaces all fonts with Bitstream Charter, Bitstream Vera Sans and
% Bitstream Vera Mono. Math will be rendered in Charter.
\usepackage[charter, greekuppercase=italicized]{mathdesign}
\usepackage{beramono}
\usepackage{berasans}

% Bold, sans-serif tensors. This fragment is taken from “egreg” from
% http://tex.stackexchange.com/a/82747/8945 and licensed under `CC-BY-SA
% <https://creativecommons.org/licenses/by-sa/3.0/>`_.
\usepackage{bm}
\DeclareMathAlphabet{\mathsfit}{\encodingdefault}{\sfdefault}{m}{sl}
\SetMathAlphabet{\mathsfit}{bold}{\encodingdefault}{\sfdefault}{bx}{sl}
\newcommand{\tens}[1]{\bm{\mathsfit{#1}}}

% Bold vectors.
\renewcommand{\vec}[1]{\boldsymbol{#1}}

%%%%%%%%%%%%%%%%%%%%%%%%%%%%%%%%%%%%%%%%%%%%%%%%%%%%%%%%%%%%%%%%%%%%%%%%%%%%%%%
%                               Input encoding                                %
%%%%%%%%%%%%%%%%%%%%%%%%%%%%%%%%%%%%%%%%%%%%%%%%%%%%%%%%%%%%%%%%%%%%%%%%%%%%%%%

\usepackage[T1]{fontenc}
\usepackage[utf8]{inputenc}

%%%%%%%%%%%%%%%%%%%%%%%%%%%%%%%%%%%%%%%%%%%%%%%%%%%%%%%%%%%%%%%%%%%%%%%%%%%%%%%
%                         Hyperrefs and PDF metadata                          %
%%%%%%%%%%%%%%%%%%%%%%%%%%%%%%%%%%%%%%%%%%%%%%%%%%%%%%%%%%%%%%%%%%%%%%%%%%%%%%%

\usepackage{hyperref}
\usepackage{lastpage}

% This sets the author in the properties of the PDF as well. If you want to
% change it, just override it with another ``\hypersetup`` call.
\hypersetup{
	breaklinks=false,
	citecolor=darkgreen,
	colorlinks=true,
	linkcolor=black,
	menucolor=black,
	pdfauthor={Martin Ueding},
	urlcolor=darkblue,
}

%%%%%%%%%%%%%%%%%%%%%%%%%%%%%%%%%%%%%%%%%%%%%%%%%%%%%%%%%%%%%%%%%%%%%%%%%%%%%%%
%                               Math Operators                                %
%%%%%%%%%%%%%%%%%%%%%%%%%%%%%%%%%%%%%%%%%%%%%%%%%%%%%%%%%%%%%%%%%%%%%%%%%%%%%%%

% AMS environments like ``align`` and theorems like ``proof``.
\usepackage{amsmath}
\usepackage{amsthm}

% Common math constructs like partial derivatives.
\usepackage{commath}

% Physical units.
\usepackage{siunitx}

% Word like operators.
\DeclareMathOperator{\acosh}{arcosh}
\DeclareMathOperator{\arcosh}{arcosh}
\DeclareMathOperator{\arcsinh}{arsinh}
\DeclareMathOperator{\arsinh}{arsinh}
\DeclareMathOperator{\asinh}{arsinh}
\DeclareMathOperator{\card}{card}
\DeclareMathOperator{\csch}{cshs}
\DeclareMathOperator{\diam}{diam}
\DeclareMathOperator{\sech}{sech}
\renewcommand{\Im}{\mathop{{}\mathrm{Im}}\nolimits}
\renewcommand{\Re}{\mathop{{}\mathrm{Re}}\nolimits}

% Fourier transform.
\DeclareMathOperator{\fourier}{\ensuremath{\mathcal{F}}}

% Roman versions of “e” and “i” to serve as Euler's number and the imaginary
% constant.
\newcommand{\ee}{\eup}
\newcommand{\eup}{\mathrm e}
\newcommand{\ii}{\iup}
\newcommand{\iup}{\mathrm i}

% Symbols for the various mathematical fields (natural numbers, integers,
% rational numbers, real numbers, complex numbers).
\newcommand{\C}{\ensuremath{\mathbb C}}
\newcommand{\N}{\ensuremath{\mathbb N}}
\newcommand{\Q}{\ensuremath{\mathbb Q}}
\newcommand{\R}{\ensuremath{\mathbb R}}
\newcommand{\Z}{\ensuremath{\mathbb Z}}

% Shape like operators.
\DeclareMathOperator{\dalambert}{\Box}
\DeclareMathOperator{\laplace}{\bigtriangleup}
\newcommand{\curl}{\vnabla \times}
\newcommand{\divergence}[1]{\inner{\vnabla}{#1}}
\newcommand{\vnabla}{\vec \nabla}

\newcommand{\half}{\frac 12}

% Unit vector (German „Einheitsvektor“).
\newcommand{\ev}{\hat{\vec e}}

% Scientific notation for large numbers.
\newcommand{\e}[1]{\cdot 10^{#1}}

% Mathematician's notation for the inner (scalar, dot) product.
\newcommand{\inner}[2]{\left\langle #1, #2 \right\rangle}

% Placeholders.
\newcommand{\emesswert}{\del{\messwert \pm \messwert}}
\newcommand{\fehlt}{\textcolor{darkred}{Hier fehlen noch Inhalte.}}
\newcommand{\messwert}{\textcolor{blue}{\square}}
\newcommand{\punkte}{\textcolor{white}{xxxxx}}

% Separator for equations on a single line.
\newcommand{\eqnsep}{,\quad}

% Quantum Mechanics
\newcommand{\bra}[1]{\left\langle #1 \right|}
\newcommand{\ket}[1]{\left| #1 \right\rangle}
\newcommand{\braket}[2]{\left\langle #1 \left. \vphantom{#1 #2} \right| #2 \right\rangle}

%%%%%%%%%%%%%%%%%%%%%%%%%%%%%%%%%%%%%%%%%%%%%%%%%%%%%%%%%%%%%%%%%%%%%%%%%%%%%%%
%                                  Headings                                   %
%%%%%%%%%%%%%%%%%%%%%%%%%%%%%%%%%%%%%%%%%%%%%%%%%%%%%%%%%%%%%%%%%%%%%%%%%%%%%%%

% This will set fancy headings to the top of the page. The page number will be
% accompanied by the total number of pages. That way, you will know if any page
% is missing.
%
% If you do not want this for your document, you can just use
% ``\pagestyle{plain}``.

\usepackage{scrpage2}

\pagestyle{scrheadings}

\automark{section}

\cfoot{\footnotesize{Seite \thepage\ / \pageref{LastPage}}}
\chead{}
\ihead{}
\ohead{\rightmark}

\setheadsepline{.4pt}

%%%%%%%%%%%%%%%%%%%%%%%%%%%%%%%%%%%%%%%%%%%%%%%%%%%%%%%%%%%%%%%%%%%%%%%%%%%%%%%
%                            Bibliography (BibTeX)                            %
%%%%%%%%%%%%%%%%%%%%%%%%%%%%%%%%%%%%%%%%%%%%%%%%%%%%%%%%%%%%%%%%%%%%%%%%%%%%%%%

\newcommand{\bibliographyfile}{../../zentrale_BibTeX/Central}

%%%%%%%%%%%%%%%%%%%%%%%%%%%%%%%%%%%%%%%%%%%%%%%%%%%%%%%%%%%%%%%%%%%%%%%%%%%%%%%
%                                Abbreviations                                %
%%%%%%%%%%%%%%%%%%%%%%%%%%%%%%%%%%%%%%%%%%%%%%%%%%%%%%%%%%%%%%%%%%%%%%%%%%%%%%%

\newcommand{\dhabk}{\mbox{d.\,h.}}
`` in your main
% document and start with ``\title`` and ``\begin{document}`` afterwards.

% If you need to add additional packages, I recommend not doing this in this
% file, but in your main document. That way, you can just drop in a new
% ``header.tex`` and get all the new commands without having to merge manually.

% Since this file encorporates a CC-BY-SA fragment, this whole files is
% licensed under the CC-BY-SA license.

\documentclass[11pt, ngerman, fleqn, DIV=15, headinclude]{scrartcl}

\usepackage{graphicx}

%%%%%%%%%%%%%%%%%%%%%%%%%%%%%%%%%%%%%%%%%%%%%%%%%%%%%%%%%%%%%%%%%%%%%%%%%%%%%%%
%                                Locale, date                                 %
%%%%%%%%%%%%%%%%%%%%%%%%%%%%%%%%%%%%%%%%%%%%%%%%%%%%%%%%%%%%%%%%%%%%%%%%%%%%%%%

\usepackage{babel}
\usepackage[iso]{isodate}

%%%%%%%%%%%%%%%%%%%%%%%%%%%%%%%%%%%%%%%%%%%%%%%%%%%%%%%%%%%%%%%%%%%%%%%%%%%%%%%
%                          Margins and other spacing                          %
%%%%%%%%%%%%%%%%%%%%%%%%%%%%%%%%%%%%%%%%%%%%%%%%%%%%%%%%%%%%%%%%%%%%%%%%%%%%%%%

\usepackage[activate]{pdfcprot}
\usepackage[parfill]{parskip}
\usepackage{setspace}

\setlength{\columnsep}{2cm}

%%%%%%%%%%%%%%%%%%%%%%%%%%%%%%%%%%%%%%%%%%%%%%%%%%%%%%%%%%%%%%%%%%%%%%%%%%%%%%%
%                                    Color                                    %
%%%%%%%%%%%%%%%%%%%%%%%%%%%%%%%%%%%%%%%%%%%%%%%%%%%%%%%%%%%%%%%%%%%%%%%%%%%%%%%

\usepackage{color}

\definecolor{darkblue}{rgb}{0,0,.5}
\definecolor{darkgreen}{rgb}{0,.5,0}
\definecolor{darkred}{rgb}{.7,0,0}

%%%%%%%%%%%%%%%%%%%%%%%%%%%%%%%%%%%%%%%%%%%%%%%%%%%%%%%%%%%%%%%%%%%%%%%%%%%%%%%
%                         Font and font like settings                         %
%%%%%%%%%%%%%%%%%%%%%%%%%%%%%%%%%%%%%%%%%%%%%%%%%%%%%%%%%%%%%%%%%%%%%%%%%%%%%%%

% This replaces all fonts with Bitstream Charter, Bitstream Vera Sans and
% Bitstream Vera Mono. Math will be rendered in Charter.
\usepackage[charter, greekuppercase=italicized]{mathdesign}
\usepackage{beramono}
\usepackage{berasans}

% Bold, sans-serif tensors. This fragment is taken from “egreg” from
% http://tex.stackexchange.com/a/82747/8945 and licensed under `CC-BY-SA
% <https://creativecommons.org/licenses/by-sa/3.0/>`_.
\usepackage{bm}
\DeclareMathAlphabet{\mathsfit}{\encodingdefault}{\sfdefault}{m}{sl}
\SetMathAlphabet{\mathsfit}{bold}{\encodingdefault}{\sfdefault}{bx}{sl}
\newcommand{\tens}[1]{\bm{\mathsfit{#1}}}

% Bold vectors.
\renewcommand{\vec}[1]{\boldsymbol{#1}}

%%%%%%%%%%%%%%%%%%%%%%%%%%%%%%%%%%%%%%%%%%%%%%%%%%%%%%%%%%%%%%%%%%%%%%%%%%%%%%%
%                               Input encoding                                %
%%%%%%%%%%%%%%%%%%%%%%%%%%%%%%%%%%%%%%%%%%%%%%%%%%%%%%%%%%%%%%%%%%%%%%%%%%%%%%%

\usepackage[T1]{fontenc}
\usepackage[utf8]{inputenc}

%%%%%%%%%%%%%%%%%%%%%%%%%%%%%%%%%%%%%%%%%%%%%%%%%%%%%%%%%%%%%%%%%%%%%%%%%%%%%%%
%                         Hyperrefs and PDF metadata                          %
%%%%%%%%%%%%%%%%%%%%%%%%%%%%%%%%%%%%%%%%%%%%%%%%%%%%%%%%%%%%%%%%%%%%%%%%%%%%%%%

\usepackage{hyperref}
\usepackage{lastpage}

% This sets the author in the properties of the PDF as well. If you want to
% change it, just override it with another ``\hypersetup`` call.
\hypersetup{
	breaklinks=false,
	citecolor=darkgreen,
	colorlinks=true,
	linkcolor=black,
	menucolor=black,
	pdfauthor={Martin Ueding},
	urlcolor=darkblue,
}

%%%%%%%%%%%%%%%%%%%%%%%%%%%%%%%%%%%%%%%%%%%%%%%%%%%%%%%%%%%%%%%%%%%%%%%%%%%%%%%
%                               Math Operators                                %
%%%%%%%%%%%%%%%%%%%%%%%%%%%%%%%%%%%%%%%%%%%%%%%%%%%%%%%%%%%%%%%%%%%%%%%%%%%%%%%

% AMS environments like ``align`` and theorems like ``proof``.
\usepackage{amsmath}
\usepackage{amsthm}

% Common math constructs like partial derivatives.
\usepackage{commath}

% Physical units.
\usepackage{siunitx}

% Word like operators.
\DeclareMathOperator{\acosh}{arcosh}
\DeclareMathOperator{\arcosh}{arcosh}
\DeclareMathOperator{\arcsinh}{arsinh}
\DeclareMathOperator{\arsinh}{arsinh}
\DeclareMathOperator{\asinh}{arsinh}
\DeclareMathOperator{\card}{card}
\DeclareMathOperator{\csch}{cshs}
\DeclareMathOperator{\diam}{diam}
\DeclareMathOperator{\sech}{sech}
\renewcommand{\Im}{\mathop{{}\mathrm{Im}}\nolimits}
\renewcommand{\Re}{\mathop{{}\mathrm{Re}}\nolimits}

% Fourier transform.
\DeclareMathOperator{\fourier}{\ensuremath{\mathcal{F}}}

% Roman versions of “e” and “i” to serve as Euler's number and the imaginary
% constant.
\newcommand{\ee}{\eup}
\newcommand{\eup}{\mathrm e}
\newcommand{\ii}{\iup}
\newcommand{\iup}{\mathrm i}

% Symbols for the various mathematical fields (natural numbers, integers,
% rational numbers, real numbers, complex numbers).
\newcommand{\C}{\ensuremath{\mathbb C}}
\newcommand{\N}{\ensuremath{\mathbb N}}
\newcommand{\Q}{\ensuremath{\mathbb Q}}
\newcommand{\R}{\ensuremath{\mathbb R}}
\newcommand{\Z}{\ensuremath{\mathbb Z}}

% Shape like operators.
\DeclareMathOperator{\dalambert}{\Box}
\DeclareMathOperator{\laplace}{\bigtriangleup}
\newcommand{\curl}{\vnabla \times}
\newcommand{\divergence}[1]{\inner{\vnabla}{#1}}
\newcommand{\vnabla}{\vec \nabla}

\newcommand{\half}{\frac 12}

% Unit vector (German „Einheitsvektor“).
\newcommand{\ev}{\hat{\vec e}}

% Scientific notation for large numbers.
\newcommand{\e}[1]{\cdot 10^{#1}}

% Mathematician's notation for the inner (scalar, dot) product.
\newcommand{\inner}[2]{\left\langle #1, #2 \right\rangle}

% Placeholders.
\newcommand{\emesswert}{\del{\messwert \pm \messwert}}
\newcommand{\fehlt}{\textcolor{darkred}{Hier fehlen noch Inhalte.}}
\newcommand{\messwert}{\textcolor{blue}{\square}}
\newcommand{\punkte}{\textcolor{white}{xxxxx}}

% Separator for equations on a single line.
\newcommand{\eqnsep}{,\quad}

% Quantum Mechanics
\newcommand{\bra}[1]{\left\langle #1 \right|}
\newcommand{\ket}[1]{\left| #1 \right\rangle}
\newcommand{\braket}[2]{\left\langle #1 \left. \vphantom{#1 #2} \right| #2 \right\rangle}

%%%%%%%%%%%%%%%%%%%%%%%%%%%%%%%%%%%%%%%%%%%%%%%%%%%%%%%%%%%%%%%%%%%%%%%%%%%%%%%
%                                  Headings                                   %
%%%%%%%%%%%%%%%%%%%%%%%%%%%%%%%%%%%%%%%%%%%%%%%%%%%%%%%%%%%%%%%%%%%%%%%%%%%%%%%

% This will set fancy headings to the top of the page. The page number will be
% accompanied by the total number of pages. That way, you will know if any page
% is missing.
%
% If you do not want this for your document, you can just use
% ``\pagestyle{plain}``.

\usepackage{scrpage2}

\pagestyle{scrheadings}

\automark{section}

\cfoot{\footnotesize{Seite \thepage\ / \pageref{LastPage}}}
\chead{}
\ihead{}
\ohead{\rightmark}

\setheadsepline{.4pt}

%%%%%%%%%%%%%%%%%%%%%%%%%%%%%%%%%%%%%%%%%%%%%%%%%%%%%%%%%%%%%%%%%%%%%%%%%%%%%%%
%                            Bibliography (BibTeX)                            %
%%%%%%%%%%%%%%%%%%%%%%%%%%%%%%%%%%%%%%%%%%%%%%%%%%%%%%%%%%%%%%%%%%%%%%%%%%%%%%%

\newcommand{\bibliographyfile}{../../zentrale_BibTeX/Central}

%%%%%%%%%%%%%%%%%%%%%%%%%%%%%%%%%%%%%%%%%%%%%%%%%%%%%%%%%%%%%%%%%%%%%%%%%%%%%%%
%                                Abbreviations                                %
%%%%%%%%%%%%%%%%%%%%%%%%%%%%%%%%%%%%%%%%%%%%%%%%%%%%%%%%%%%%%%%%%%%%%%%%%%%%%%%

\newcommand{\dhabk}{\mbox{d.\,h.}}
`` in your main
% document and start with ``\title`` and ``\begin{document}`` afterwards.

% If you need to add additional packages, I recommend not doing this in this
% file, but in your main document. That way, you can just drop in a new
% ``header.tex`` and get all the new commands without having to merge manually.

% Since this file encorporates a CC-BY-SA fragment, this whole files is
% licensed under the CC-BY-SA license.

\documentclass[11pt, ngerman, fleqn, DIV=15, headinclude]{scrartcl}

\usepackage{graphicx}

%%%%%%%%%%%%%%%%%%%%%%%%%%%%%%%%%%%%%%%%%%%%%%%%%%%%%%%%%%%%%%%%%%%%%%%%%%%%%%%
%                                Locale, date                                 %
%%%%%%%%%%%%%%%%%%%%%%%%%%%%%%%%%%%%%%%%%%%%%%%%%%%%%%%%%%%%%%%%%%%%%%%%%%%%%%%

\usepackage{babel}
\usepackage[iso]{isodate}

%%%%%%%%%%%%%%%%%%%%%%%%%%%%%%%%%%%%%%%%%%%%%%%%%%%%%%%%%%%%%%%%%%%%%%%%%%%%%%%
%                          Margins and other spacing                          %
%%%%%%%%%%%%%%%%%%%%%%%%%%%%%%%%%%%%%%%%%%%%%%%%%%%%%%%%%%%%%%%%%%%%%%%%%%%%%%%

\usepackage[activate]{pdfcprot}
\usepackage[parfill]{parskip}
\usepackage{setspace}

\setlength{\columnsep}{2cm}

%%%%%%%%%%%%%%%%%%%%%%%%%%%%%%%%%%%%%%%%%%%%%%%%%%%%%%%%%%%%%%%%%%%%%%%%%%%%%%%
%                                    Color                                    %
%%%%%%%%%%%%%%%%%%%%%%%%%%%%%%%%%%%%%%%%%%%%%%%%%%%%%%%%%%%%%%%%%%%%%%%%%%%%%%%

\usepackage{color}

\definecolor{darkblue}{rgb}{0,0,.5}
\definecolor{darkgreen}{rgb}{0,.5,0}
\definecolor{darkred}{rgb}{.7,0,0}

%%%%%%%%%%%%%%%%%%%%%%%%%%%%%%%%%%%%%%%%%%%%%%%%%%%%%%%%%%%%%%%%%%%%%%%%%%%%%%%
%                         Font and font like settings                         %
%%%%%%%%%%%%%%%%%%%%%%%%%%%%%%%%%%%%%%%%%%%%%%%%%%%%%%%%%%%%%%%%%%%%%%%%%%%%%%%

% This replaces all fonts with Bitstream Charter, Bitstream Vera Sans and
% Bitstream Vera Mono. Math will be rendered in Charter.
\usepackage[charter, greekuppercase=italicized]{mathdesign}
\usepackage{beramono}
\usepackage{berasans}

% Bold, sans-serif tensors. This fragment is taken from “egreg” from
% http://tex.stackexchange.com/a/82747/8945 and licensed under `CC-BY-SA
% <https://creativecommons.org/licenses/by-sa/3.0/>`_.
\usepackage{bm}
\DeclareMathAlphabet{\mathsfit}{\encodingdefault}{\sfdefault}{m}{sl}
\SetMathAlphabet{\mathsfit}{bold}{\encodingdefault}{\sfdefault}{bx}{sl}
\newcommand{\tens}[1]{\bm{\mathsfit{#1}}}

% Bold vectors.
\renewcommand{\vec}[1]{\boldsymbol{#1}}

%%%%%%%%%%%%%%%%%%%%%%%%%%%%%%%%%%%%%%%%%%%%%%%%%%%%%%%%%%%%%%%%%%%%%%%%%%%%%%%
%                               Input encoding                                %
%%%%%%%%%%%%%%%%%%%%%%%%%%%%%%%%%%%%%%%%%%%%%%%%%%%%%%%%%%%%%%%%%%%%%%%%%%%%%%%

\usepackage[T1]{fontenc}
\usepackage[utf8]{inputenc}

%%%%%%%%%%%%%%%%%%%%%%%%%%%%%%%%%%%%%%%%%%%%%%%%%%%%%%%%%%%%%%%%%%%%%%%%%%%%%%%
%                         Hyperrefs and PDF metadata                          %
%%%%%%%%%%%%%%%%%%%%%%%%%%%%%%%%%%%%%%%%%%%%%%%%%%%%%%%%%%%%%%%%%%%%%%%%%%%%%%%

\usepackage{hyperref}
\usepackage{lastpage}

% This sets the author in the properties of the PDF as well. If you want to
% change it, just override it with another ``\hypersetup`` call.
\hypersetup{
	breaklinks=false,
	citecolor=darkgreen,
	colorlinks=true,
	linkcolor=black,
	menucolor=black,
	pdfauthor={Martin Ueding},
	urlcolor=darkblue,
}

%%%%%%%%%%%%%%%%%%%%%%%%%%%%%%%%%%%%%%%%%%%%%%%%%%%%%%%%%%%%%%%%%%%%%%%%%%%%%%%
%                               Math Operators                                %
%%%%%%%%%%%%%%%%%%%%%%%%%%%%%%%%%%%%%%%%%%%%%%%%%%%%%%%%%%%%%%%%%%%%%%%%%%%%%%%

% AMS environments like ``align`` and theorems like ``proof``.
\usepackage{amsmath}
\usepackage{amsthm}

% Common math constructs like partial derivatives.
\usepackage{commath}

% Physical units.
\usepackage{siunitx}

% Word like operators.
\DeclareMathOperator{\acosh}{arcosh}
\DeclareMathOperator{\arcosh}{arcosh}
\DeclareMathOperator{\arcsinh}{arsinh}
\DeclareMathOperator{\arsinh}{arsinh}
\DeclareMathOperator{\asinh}{arsinh}
\DeclareMathOperator{\card}{card}
\DeclareMathOperator{\csch}{cshs}
\DeclareMathOperator{\diam}{diam}
\DeclareMathOperator{\sech}{sech}
\renewcommand{\Im}{\mathop{{}\mathrm{Im}}\nolimits}
\renewcommand{\Re}{\mathop{{}\mathrm{Re}}\nolimits}

% Fourier transform.
\DeclareMathOperator{\fourier}{\ensuremath{\mathcal{F}}}

% Roman versions of “e” and “i” to serve as Euler's number and the imaginary
% constant.
\newcommand{\ee}{\eup}
\newcommand{\eup}{\mathrm e}
\newcommand{\ii}{\iup}
\newcommand{\iup}{\mathrm i}

% Symbols for the various mathematical fields (natural numbers, integers,
% rational numbers, real numbers, complex numbers).
\newcommand{\C}{\ensuremath{\mathbb C}}
\newcommand{\N}{\ensuremath{\mathbb N}}
\newcommand{\Q}{\ensuremath{\mathbb Q}}
\newcommand{\R}{\ensuremath{\mathbb R}}
\newcommand{\Z}{\ensuremath{\mathbb Z}}

% Shape like operators.
\DeclareMathOperator{\dalambert}{\Box}
\DeclareMathOperator{\laplace}{\bigtriangleup}
\newcommand{\curl}{\vnabla \times}
\newcommand{\divergence}[1]{\inner{\vnabla}{#1}}
\newcommand{\vnabla}{\vec \nabla}

\newcommand{\half}{\frac 12}

% Unit vector (German „Einheitsvektor“).
\newcommand{\ev}{\hat{\vec e}}

% Scientific notation for large numbers.
\newcommand{\e}[1]{\cdot 10^{#1}}

% Mathematician's notation for the inner (scalar, dot) product.
\newcommand{\inner}[2]{\left\langle #1, #2 \right\rangle}

% Placeholders.
\newcommand{\emesswert}{\del{\messwert \pm \messwert}}
\newcommand{\fehlt}{\textcolor{darkred}{Hier fehlen noch Inhalte.}}
\newcommand{\messwert}{\textcolor{blue}{\square}}
\newcommand{\punkte}{\textcolor{white}{xxxxx}}

% Separator for equations on a single line.
\newcommand{\eqnsep}{,\quad}

% Quantum Mechanics
\newcommand{\bra}[1]{\left\langle #1 \right|}
\newcommand{\ket}[1]{\left| #1 \right\rangle}
\newcommand{\braket}[2]{\left\langle #1 \left. \vphantom{#1 #2} \right| #2 \right\rangle}

%%%%%%%%%%%%%%%%%%%%%%%%%%%%%%%%%%%%%%%%%%%%%%%%%%%%%%%%%%%%%%%%%%%%%%%%%%%%%%%
%                                  Headings                                   %
%%%%%%%%%%%%%%%%%%%%%%%%%%%%%%%%%%%%%%%%%%%%%%%%%%%%%%%%%%%%%%%%%%%%%%%%%%%%%%%

% This will set fancy headings to the top of the page. The page number will be
% accompanied by the total number of pages. That way, you will know if any page
% is missing.
%
% If you do not want this for your document, you can just use
% ``\pagestyle{plain}``.

\usepackage{scrpage2}

\pagestyle{scrheadings}

\automark{section}

\cfoot{\footnotesize{Seite \thepage\ / \pageref{LastPage}}}
\chead{}
\ihead{}
\ohead{\rightmark}

\setheadsepline{.4pt}

%%%%%%%%%%%%%%%%%%%%%%%%%%%%%%%%%%%%%%%%%%%%%%%%%%%%%%%%%%%%%%%%%%%%%%%%%%%%%%%
%                            Bibliography (BibTeX)                            %
%%%%%%%%%%%%%%%%%%%%%%%%%%%%%%%%%%%%%%%%%%%%%%%%%%%%%%%%%%%%%%%%%%%%%%%%%%%%%%%

\newcommand{\bibliographyfile}{../../zentrale_BibTeX/Central}

%%%%%%%%%%%%%%%%%%%%%%%%%%%%%%%%%%%%%%%%%%%%%%%%%%%%%%%%%%%%%%%%%%%%%%%%%%%%%%%
%                                Abbreviations                                %
%%%%%%%%%%%%%%%%%%%%%%%%%%%%%%%%%%%%%%%%%%%%%%%%%%%%%%%%%%%%%%%%%%%%%%%%%%%%%%%

\newcommand{\dhabk}{\mbox{d.\,h.}}


\usepackage{cancel}
\usepackage{tikz}

\newcommand{\themodul}{physik421}
\newcommand{\thegruppe}{Gruppe 4 -- Franz Niecknig}
\newcommand{\theuebung}{6}

\ifoot{\footnotesize{Martin Ueding}}
\ihead{\themodul{} -- Übung \theuebung}
\ofoot{\footnotesize{\thegruppe}}

\def\thesection{H \arabic{section}}
\def\thesubsection{\thesection\alph{subsection}}

\setcounter{section}{7}

\title{\themodul{} -- Übung \theuebung}
\subtitle{\thegruppe}
\author{
	Martin Ueding \footnote{\href{mailto:mu@uni-bonn.de}{mu@uni-bonn.de}}
}

\hypersetup{
	pdftitle={\themodul{} - Übung \theuebung},
}

\begin{document}

\maketitle

\begin{center}
	\ccbysadetitle
\end{center}

\begin{table}[h]
	\centering
	\begin{tabular}{l|c|c|c}
		Aufgabe
		& \ref 1
		& \ref 2
		& $\sum$ \\
		\hline
		Punkte
		& \punkte / 9
		& \punkte / 11
		& \punkte / 20
	\end{tabular}
\end{table}

%%%%%%%%%%%%%%%%%%%%%%%%%%%%%%%%%%%%%%%%%%%%%%%%%%%%%%%%%%%%%%%%%%%%%%%%%%%%%%%
%                    Unschärfe der kohärenten Zustände                     %
%%%%%%%%%%%%%%%%%%%%%%%%%%%%%%%%%%%%%%%%%%%%%%%%%%%%%%%%%%%%%%%%%%%%%%%%%%%%%%%

\section{Unschärfe der kohärenten Zustände}
\label 1

\subsection{Erwartungswert des Ortsoperators}

\begin{align*}
	\braopket{\phi}{\hat x}{\phi}
	&= \sqrt{\frac{\hbar}{2m\omega}} \braopket\phi{\hat a + \hat a^\dagger}\phi \\
	&= \sqrt{\frac{\hbar}{2m\omega}} \del{\braket\phi{\hat a \phi} + \braket{\hat a\phi}\phi} \\
	\intertext{%
		Benutze komplexe Symmetrie des Skalarprodukts.
	}
	&= \sqrt{\frac{\hbar}{2m\omega}} \del{\braket\phi{\hat a \phi} + \braket{\phi}{\hat a\phi}^*} \\
	\intertext{%
		Benutze gegebene Eigenwertgleichung für den Absteigeoperator.
	}
	&= \sqrt{\frac{\hbar}{2m\omega}} \del{\alpha + \alpha^*}
\end{align*}

\begin{align*}
	\braopket{\phi}{\hat x^2}{\phi}
	&= \frac{\hbar}{2m\omega} \braopket\phi{\del{\hat a + \hat a^\dagger}^2}\phi \\
	&= \frac{\hbar}{2m\omega} \braopket\phi{\hat a^2 + \hat a \hat a^\dagger + \hat a^\dagger \hat a + \hat a^{\dagger2}}\phi \\
	\intertext{%
		Benutze den Kommutator $\sbr{\hat a, \hat a^\dagger} = 1$.
	}
	&= \frac{\hbar}{2m\omega} \braopket\phi{\hat a^2 + 2\hat a^\dagger \hat a + \hat a^{\dagger2} + 1}\phi \\
	&= \frac{\hbar}{2m\omega} \del{\alpha^2 + 2\braopket\phi{\hat a^\dagger \hat a}\phi + \alpha^{*2} + 1} \\
	&= \frac{\hbar}{2m\omega} \del{\alpha^2 + 2\braket{\hat a\phi}{\hat a\phi} + \alpha^{*2} + 1} \\
	&= \frac{\hbar}{2m\omega} \del{\alpha^2 + 2 \alpha \alpha^* + \alpha^{*2} + 1} \\
	&= \frac{\hbar}{2m\omega} \del{\del{\alpha+ \alpha^*}^2 + 1}
\end{align*}

Die Ortsunschärfe ist die Differenz von $\bracket{x^2}$ und $\bracket{x}^2$. Also:
\[
	\del{\Deltaup x}^2 = \frac{\hbar}{2m\omega} 
\]

\subsection{Impulsunschärfe}

Gehe analog für den Impuls vor. Dort ist:
\[
	\hat p = \frac 1i \sqrt{\frac{\hbar m \omega}{2}} \del{\hat a - \hat a^\dagger}
\]

Dadurch ändert sich der Vorfaktor und die Vorzeichen in den Mischtermen:
\[
	\bracket{\hat p} =  \frac 1i \sqrt{\frac{\hbar m \omega}{2}} \del{\alpha + \alpha^*}
\]
\[
	\bracket{\hat p^2} = - \frac{\hbar m \omega}{2} \del{\del{\alpha - \alpha^*}^2 - 1}
\]

Somit ist die Impulsunschärfe:
\[
	\del{\Deltaup p}^2 = \frac{\hbar m \omega}{2}
\]

Und deren Produkt:
\[
	\sqrt{
		\frac{\hbar}{2m\omega} 
		\frac{\hbar m \omega}{2}
	}
	= \frac \hbar2
\]

\subsection{Energie und -unschärfe}

Mit dem Hamiltonoperator:
\[
	\hat H
	= \frac{\hbar\omega}2 \del{\hat a \hat a^\dagger + \hat a^\dagger \hat a}
	= \frac{\hbar\omega}2 \del{2 \hat a^\dagger \hat a + 1}
\]

Berechne Energieerwartungswert:
\begin{align*}
	\braopket{\phi}{\hat H}{\phi}
	&= \frac{\hbar\omega}{2} \braopket{\phi}{2 \hat a^\dagger \hat a + 1}{\phi} \\
	&= \frac{\hbar\omega}2 \del{2 \alpha^* \alpha + 1}
\end{align*}

Berechne zweites Moment der Verteilung:
\begin{align*}
	\braopket{\phi}{\hat H^2}{\phi}
	&= \frac{\hbar^2\omega^2}{4} \braopket{\phi}{\del{2 \hat a^\dagger \hat a + 1^2}}{\phi} \\
	&= \frac{\hbar^2\omega^2}{4} \braopket{\phi}{4 \hat a^\dagger \hat a \hat a^\dagger \hat a + 4 \hat a^\dagger \hat a + 1}{\phi} \\
	&= \frac{\hbar^2\omega^2}{4} \del{\braopket{\phi}{4 \hat a^\dagger \hat a \hat a^\dagger \hat a}{\phi} + 4 \alpha^* \alpha + 1} \\
	&= \frac{\hbar^2\omega^2}{4} \del{4 \braopket{\hat a \phi}{\hat a \hat a^\dagger}{\hat a \phi} + 4 \alpha^* \alpha + 1} \\
	&= \frac{\hbar^2\omega^2}{4} \del{4 \alpha^* \alpha \braopket{\phi}{\hat a \hat a^\dagger}{\phi} + 4 \alpha^* \alpha + 1} \\
	\intertext{%
		Nutze den Kommutator.
	}
	&= \frac{\hbar^2\omega^2}{4} \del{4 \alpha^* \alpha \braopket{\phi}{\hat a^\dagger \hat a + 1}{\phi} + 4 \alpha^* \alpha + 1} \\
	&= \frac{\hbar^2\omega^2}{4} \del{4 \alpha^* \alpha \del{\alpha^* \alpha + 1} + 4 \alpha^* \alpha + 1} \\
	&= \frac{\hbar^2\omega^2}{4} \del{4 \alpha^{*2} \alpha^2 + 8 \alpha^* \alpha + 1} \\
	&= \frac{\hbar^2\omega^2}{4} \del{\del{2\alpha^* \alpha + 1}^2 + 4 \alpha^* \alpha}
\end{align*}

Die Differenz ist:
\[
	\del{\Deltaup \hat H}^2 = \hbar \omega \abs\alpha^2
\]

Breite des Peaks:
\[
	\frac{\Deltaup \hat H}{\bracket{\hat H}}
	= \frac{\hbar \omega \abs\alpha^2}{\frac{\hbar\omega}2 \del{2 \alpha^* \alpha + 1}}
	\approx \frac{1}{\abs\alpha}
\]

%%%%%%%%%%%%%%%%%%%%%%%%%%%%%%%%%%%%%%%%%%%%%%%%%%%%%%%%%%%%%%%%%%%%%%%%%%%%%%%
%                 Differentialoperatoren in Kugelkoordinaten                  %
%%%%%%%%%%%%%%%%%%%%%%%%%%%%%%%%%%%%%%%%%%%%%%%%%%%%%%%%%%%%%%%%%%%%%%%%%%%%%%%

\section{Differentialoperatoren in Kugelkoordinaten}
\label 2

\subsection{Orthonormalität}

Zu zeigen, dass die Vektoren paarweise orthogonal sind. Die Rechnungen sind
trivial:
\begin{align*}
	\inner{\ev_r}{\ev_\theta}
	&= \inner{
	\begin{pmatrix}
		\sin\theta \cos\phi \\
		\sin\theta \sin\phi \\
		\cos\theta
	\end{pmatrix}
}{
	\begin{pmatrix}
		\cos\theta \cos\phi \\
		\cos\theta \sin\phi \\
		- \sin\theta
	\end{pmatrix}
} \\
&= \sin\theta \cos\theta - \cos\theta \sin\theta \\
&= 0
\end{align*}
\begin{align*}
	\inner{\ev_\theta}{\ev_\phi}
	&= \inner{
	\begin{pmatrix}
		\cos\theta \cos\phi \\
		\cos\theta \sin\phi \\
		- \sin\theta
	\end{pmatrix}
}{
	\begin{pmatrix}
		-\sin\phi \\
		\cos\phi \\
		0
	\end{pmatrix}
} \\
&= 0
\end{align*}
\begin{align*}
	\inner{\ev_\phi}{\ev_r}
	&= \inner{
	\begin{pmatrix}
		-\sin\phi \\
		\cos\phi \\
		0
	\end{pmatrix}
}{
	\begin{pmatrix}
		\sin\theta \cos\phi \\
		\sin\theta \sin\phi \\
		\cos\theta
	\end{pmatrix}
} \\
&= 0
\end{align*}

Der Betrag der Vektoren muss eins sein. Dies ist mit $\sin^2 + \cos^2 = 1$
schnell zu sehen.

\subsection{Gradient}

Betrachte die Ableitungen nach den Koordinaten $r, \theta, \phi$ in
Abhängigkeit von den Koordinaten $x, y, z$. 
\begin{align*}
	\dpd{}r
	&= \dpd xr \dpd{}x + \dpd yr \dpd{}y + \dpd zr \dpd{}z \\
	&= \sin\theta \cos\phi \dpd{}x + \sin\theta \sin\phi \dpd{}y + \cos\theta \dpd{}z \\
	&= \inner{\ev_r}\vnabla
	%
	\displaybreak[0] \\
	%
	\dpd{}\theta
	&= \dpd x\theta \dpd{}x + \dpd y\theta \dpd{}y + \dpd z\theta \dpd{}z \\
	&= r \cos\theta \cos\phi \dpd{}x + r \cos\theta \sin\phi \dpd{}y - r \sin\theta \dpd{}z \\
	&= r \inner{\ev_\theta}\vnabla
	%
	\displaybreak[0] \\
	%
	\dpd{}\phi
	&= \dpd x\phi \dpd{}x + \dpd y\phi \dpd {}y + \dpd z\phi \dpd{}z \\
	&= - r \sin\theta \sin\phi \dpd{}x + r \sin\theta \cos\phi \dpd{}y \\
	&= r \sin\theta \inner{\ev_\phi}\vnabla
\end{align*}

Dies liefert die Projektionen des $\vnabla$-Operators auf die Einheitsvektoren
der Kugelkoordinaten. Stelle obige Gleichungen um und erhalte:
\begin{align*}
	\nabla_r &= \dpd{}r \\
	\nabla_\theta &= \frac 1r \dpd{}\theta \\
	\nabla_\phi &= \frac1{r \sin\theta} \dpd{}\phi
\end{align*}

\subsection{Divergenz}

Die Divergenz ist wie immer $\divergence{\vec v}$. Allerdings ist $\vnabla$ nun
nicht mehr der Vektor der gradlinigen partiellen Ableitungen $\partial$,
sondern der Vektor der „covariant derivatives“. Außerdem ist der metrische
Tensor nicht mehr trivial. Betrachte folgendes als Skalarprodukt:
\[
	\sum_{i, j} \inner{\ev_i \nabla_i}{v_j, \ev_j}
\]

In einem normalen Koordinatensystem verschwinden aufgrund der Orthogonalität
der $\ev_i$ und $\ev_j$ alle Terme mit $i \neq j$. Hier treten solche Terme
allerdings auf:
\begin{align*}
	\ev_r \nabla_r v_r \ev_r
	&= \ev_r \dpd{}r v_r \ev_r \\
	&= \ev_r \dpd{v_r}r \ev_r + \ev_r v_r \underbrace{\dpd{\ev_r}{r}}_{=0} \\
	&= \dpd{v_r}r 
	%
	\displaybreak[0] \\
	%
	\ev_r \nabla_r v_\theta \ev_\theta
	&= \ev_r \dpd{}r v_\theta \ev_\theta \\
	&= \ev_r \dpd{v_r}r \ev_\theta + \ev_r v_r \underbrace{\dpd{\ev_\theta}{r}}_{=0} \\
	&= 0
	%
	\displaybreak[0] \\
	%
	\ev_r \nabla_r v_\phi \ev_\phi
	&= 0 & \text{Weil $\inner{\ev_r}{\ev_\theta} = 0$ und $\pd{\ev\phi}r = 0$.}
	%
	\displaybreak[0] \\
	%
	\ev_\theta \nabla_\theta v_r \ev_r
	&= \ev_\theta \frac 1r \dpd{}\theta v_r \ev_r \\
	&= \ev_\theta \frac 1r \dpd{v_r}\theta \ev_r + \ev_\theta \frac 1r v_r \underbrace{\dpd{\ev_r}\theta}_{=\ev_\theta} \\
	&= \frac 1r v_r
	%
	\displaybreak[0] \\
	%
	\ev_\theta \nabla_\theta v_\theta \ev_\theta
	&= \ev_\theta \frac 1r \dpd{}\theta v_\theta \ev_\theta \\
	&= \ev_\theta \frac 1r \dpd{v_\theta}\theta \ev_\theta + \underbrace{\ev_\theta \frac 1r v_\theta \underbrace{\dpd{\ev_\theta}\theta}_{=-\ev_r}}_{=0} \\
	&= \frac 1r \dpd{v_\theta}\theta
	%
	\displaybreak[0] \\
	%
	\ev_\theta \nabla_\theta v_\phi \ev_\phi
	&= 0
	%
	\displaybreak[0] \\
	%
	\ev_\phi \nabla_\phi v_r \ev_r
	&= \ev_\phi \frac1{1\sin\theta} v_r \underbrace{\dpd{\ev_r}\phi}_{=\sin\theta \ev_\phi} \\
	&= \frac 1r v_r
	%
	\displaybreak[0] \\
	%
	\ev_\phi \nabla_\phi v_\theta \ev_\theta
	&= \ev_\phi \frac1{r \sin\theta} v_\theta \underbrace{\dpd{\ev_\theta}\phi}_{=\cos\theta \ev_\phi} \\
	&= \frac 1r \tan\theta v_\theta
	%
	\displaybreak[0] \\
	%
	\ev_\phi \nabla_\phi v_\phi \ev_\phi
	&= \frac1{r\sin\theta} \dpd{v_\phi}\phi + \ev_\phi \frac1{r \sin\theta} v_\phi \dpd{\ev_\phi}\phi \\
	&= \frac1{r\sin\theta} \dpd{v_\phi}\phi
\end{align*}

Auf dem Zettel ist gegeben:
\begin{align*}
	\frac1{r^2} \dpd{}r \del{r^2 v_r} &= \dpd{v_r}r + \frac 2r v_r \\
	\frac1{r\sin\theta} \dpd{}\theta \sin\theta v_\theta &= \frac 1r \dpd{v_\theta}\theta + \frac 1r \tan\theta v_\theta \\
	\frac1{r\sin\theta} \dpd{r_\phi}\phi &
\end{align*}

Die Summanden sind alle, bis auf dem Faktor 2, oben zu finden. Somit ist die
Identität gezeigt.

\subsection{Laplaceoperator}

Der Laplaceoperator ist $\inner\vnabla\vnabla$. Oder die Divergenz des
Gradienten. Setze den Gradient in die Divergenz ein:
\[
	\frac1{r^2} \dpd{}r r^2 \dpd{}r + \frac1{r\sin\theta} \dpd{}\theta
	\sin\theta \frac 1r \dpd{}\theta + \frac1{r\sin\theta} \dpd{}\phi
	\frac1{r\sin\theta} \dpd{}\phi
\]

Ziehe alles vor die Ableitung und erhalte den gesuchten Ausdruck.

%%%%%%%%%%%%%%%%%%%%%%%%%%%%%%%%%%%%%%%%%%%%%%%%%%%%%%%%%%%%%%%%%%%%%%%%%%%%%%%
%                                    Ende                                     %
%%%%%%%%%%%%%%%%%%%%%%%%%%%%%%%%%%%%%%%%%%%%%%%%%%%%%%%%%%%%%%%%%%%%%%%%%%%%%%%

\IfFileExists{\bibliographyfile}{
	%\bibliography{\bibliographyfile}
}{}

\end{document}

% vim: spell spelllang=de
