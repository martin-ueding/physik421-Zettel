 % Copyright © 2012-2013 Martin Ueding <dev@martin-ueding.de>
%
\input{header.tex}

\usepackage{cancel}
\usepackage{tikz}
\usepackage{pdfpages}

\newcommand{\themodul}{physik421}
\newcommand{\thegruppe}{Gruppe 4 -- Franz Niecknig}
\newcommand{\theuebung}{11}

\ifoot{\footnotesize{Martin Ueding}}
\ihead{\themodul{} -- Übung \theuebung}
\ofoot{\footnotesize{\thegruppe}}

\def\thesection{H \arabic{section}}
\def\thesubsection{\thesection\alph{subsection}}

\setcounter{section}{17}

\title{\themodul{} -- Übung \theuebung}
\subtitle{\thegruppe}
\author{
	Martin Ueding \footnote{\href{mailto:mu@uni-bonn.de}{mu@uni-bonn.de}}
}

\hypersetup{
	pdftitle={\themodul{} - Übung \theuebung},
}

\begin{document}

\maketitle

\begin{center}
	\ccbysadetitle
\end{center}

%%%%%%%%%%%%%%%%%%%%%%%%%%%%%%%%%%%%%%%%%%%%%%%%%%%%%%%%%%%%%%%%%%%%%%%%%%%%%%%
%                              Der Stark-Effekt                               %
%%%%%%%%%%%%%%%%%%%%%%%%%%%%%%%%%%%%%%%%%%%%%%%%%%%%%%%%%%%%%%%%%%%%%%%%%%%%%%%

\section{Der Stark-Effekt}

\subsection{Störterm, Dipolmoment}

\subsection{Störterm in Kugelkoordinaten}

\subsection{Verschwinden der Matrixelemente}

\subsection{Verschiebung ab Ordnung}

\subsection{Diagonalisieren des Störterms}

\subsection{Erwartungswerte}

\includepdf[pages=-]{Mathematica.pdf}

%%%%%%%%%%%%%%%%%%%%%%%%%%%%%%%%%%%%%%%%%%%%%%%%%%%%%%%%%%%%%%%%%%%%%%%%%%%%%%%
%                                    Ende                                     %
%%%%%%%%%%%%%%%%%%%%%%%%%%%%%%%%%%%%%%%%%%%%%%%%%%%%%%%%%%%%%%%%%%%%%%%%%%%%%%%

\IfFileExists{\bibliographyfile}{
	%\bibliography{\bibliographyfile}
}{}

\end{document}

% vim: spell spelllang=de
