 % Copyright © 2012-2013 Martin Ueding <dev@martin-ueding.de>
%
\input{header.tex}

\usepackage{cancel}
\usepackage{tikz}

\newcommand{\themodul}{physik421}
\newcommand{\thegruppe}{Gruppe 4 -- Franz Niecknig}
\newcommand{\theuebung}{8}

\ifoot{\footnotesize{Martin Ueding}}
\ihead{\themodul{} -- Übung \theuebung}
\ofoot{\footnotesize{\thegruppe}}

\def\thesection{H \arabic{section}}
\def\thesubsection{\thesection\alph{subsection}}

\setcounter{section}{11}

\title{\themodul{} -- Übung \theuebung}
\subtitle{\thegruppe}
\author{
	Martin Ueding \footnote{\href{mailto:mu@uni-bonn.de}{mu@uni-bonn.de}}
}

\hypersetup{
	pdftitle={\themodul{} - Übung \theuebung},
}

\setcounter{secnumdepth}{2}

\begin{document}

\maketitle

\begin{center}
	\ccbysadetitle
\end{center}

%%%%%%%%%%%%%%%%%%%%%%%%%%%%%%%%%%%%%%%%%%%%%%%%%%%%%%%%%%%%%%%%%%%%%%%%%%%%%%%
%                        Nachtrag zum Wasserstoffatom                         %
%%%%%%%%%%%%%%%%%%%%%%%%%%%%%%%%%%%%%%%%%%%%%%%%%%%%%%%%%%%%%%%%%%%%%%%%%%%%%%%

\section{Nachtrag zum Wasserstoffatom}

\subsection{Ansatz}

\newcommand\er{\ee^{-\rho}}
\newcommand\omr{\omega(\rho)}

\begin{gather*}
	\del{\dpd[2]{}\rho - \frac{l(l+1)}{\rho^2} + \frac{\rho_0}\rho - 1}
	\rho^{l+1} \ee^{-\rho} \omega(\rho) = 0 \\
	%
	\begin{split}
		\dpd{}\rho \del{l(l+1) \rho^l \ee^{-\rho} \omega(\rho)
		- \rho^{l+1} \ee^{-\rho} \omega(\rho)
	+ \rho^{l+1} \ee^{-\rho} \omega'(\rho)}
	- l(l+1) \rho^{l-1} \ee^{-\rho} \omega(\rho) & \\
		+ \rho_0 \rho^l \ee^{-\rho} \omega(\rho)
		- \rho^{l+1} \ee^{-\rho} \omega(\rho)
		&= 0
	\end{split} \\
	%
	\intertext{%
		Führe an dieser Stelle Farben ein, um das Termmemory zu erleichtern.
	}
	%
	\begin{split}
		\textcolor{blue}{ l(l+1) \rho^{l-1} \er \omr }
		- \textcolor{gray}{ (l+1) \rho^l \er \omr }
		+ \textcolor{red}{ (l+1) \rho^l \er \omega'(\rho) }
		+ \textcolor{gray}{ (l+1) \rho^l \er \omr } & \\
		+ \textcolor{ForestGreen}{ \rho^{l+1} \er \omr }
		- \textcolor{orange}{ \rho^{l+1} \er \omega'(\rho) }
		+ \textcolor{red}{ (l+1) \rho^l \er \omr }
		- \textcolor{orange}{ \rho^{l+1} \er \omega'(\rho) } & \\
		+ \textcolor{cyan}{ \rho^{l+1} \er \omega''(\rho) }
		- \textcolor{blue}{ l(l+1) \rho^{l-1} \er \omr }
		+ \textcolor{green}{ \rho_0 \rho^l \er \omr }
		- \textcolor{ForestGreen}{ \rho^{l+1} \er \omr }
		&= 0
	\end{split}
	%
	\intertext{%
		Nach Umsortieren bleiben:
	}
	%
	\begin{split}
		- 2 \textcolor{gray}{ (l+1) \rho^l \er \omr }
		+ 2 \textcolor{red}{ (l+1) \rho^l \er \omega'(\rho) }
		- 2 \textcolor{orange}{ \rho^{l+1} \er \omega'(\rho) }
		+ \textcolor{cyan}{ \rho^{l+1} \er \omega''(\rho) } & \\
		+ \textcolor{green}{ \rho_0 \rho^l \er \omr }
		&= 0
	\end{split}
	%
	\intertext{%
		Entferne die $\er$ komplett, und im nächsten Schritt teile ich durch
		$\rho^l$.
	}
	%
	- 2  (l+1) \rho^l \omr
	+ 2  (l+1) \rho^l \omega'(\rho)
	- 2  \rho^{l+1} \omega'(\rho)
	+  \rho^{l+1} \omega''(\rho) 
	+ \rho_0 \rho^l \omr
	= 0 \\
	%
	- 2  (l+1) \omr
	+ 2  (l+1) \omega'(\rho)
	- 2  \rho \omega'(\rho)
	+  \rho \omega''(\rho) 
	+ \rho_0 \omr
	= 0
	%
	\intertext{%
		Umstellen und Gruppieren gibt die gewünschte Differentialgleichung.
	}
	%
	\rho \omega''(\rho) + 2(l+1-\rho) \omega'(\rho) + \del{\rho_0 - 2(l+1)} \omega(\rho) = 0
\end{gather*}

\subsection{Potenzreihenansatz}

Setze an mit:
\[
	\omega(\rho) = \sum_k a_k \rho^k
\]

Setze dies in die Differentialgleichung ein:
\[
	\sum_k \del{k(k-1) a_k \rho^{k-1} + 2(l+1-\rho) k a_k \rho^{k-1} + \del{\rho_0 - 2(l+1)} a_k \rho^k} = 0
\]

Dies ist auf jeden Fall erfüllt, wenn jeder Summand null ist. Also:
\begin{align*}
	(k+1) k a_{k+1} \rho^k + 2(l+1)(k+1) a_{k+1} \rho^k
	- 2(k+1) a_{k+1} \rho^{k+1} + \del{\rho_0 - 2(l+1)} a_k \rho^k &= 0 \\
	%
	(k+1) k a_{k+1} + 2(l+1)(k+1) a_{k+1} - 2k a_k
	+ \del{\rho_0 - 2(l+1)} a_k &= 0 \\
	%
	\del{(k+1) k + 2(l+1)(k+1) } a_{k+1}
	+ \del{\rho_0 - 2(l+1) - 2k} a_k &= 0
\end{align*}

Erhalte die gesuchte Rekursionsbeziehung:
\[
	a_{k+1} = \frac{2(l+1) + 2k - \rho_0}{(k+1)(k+2l+2)} a_k
\]

%%%%%%%%%%%%%%%%%%%%%%%%%%%%%%%%%%%%%%%%%%%%%%%%%%%%%%%%%%%%%%%%%%%%%%%%%%%%%%%
%                     Erwartungswerte zum Wasserstoffatom                     %
%%%%%%%%%%%%%%%%%%%%%%%%%%%%%%%%%%%%%%%%%%%%%%%%%%%%%%%%%%%%%%%%%%%%%%%%%%%%%%%

\section{Erwartungswerte zum Wasserstoffatom}

\subsection{Reskalierung}

Beginne mit:
\begin{align*}
	\del{\dod[2]{}\rho - \frac{l(l+1)}{\rho^2} + \frac{\rho_0}\rho - 1} u_l(\rho) &= 0 \\
	\intertext{
		Substituiere $\bar\rho := \rho n$ mit $\rho = \bar\rho / n$. Nach der
		Kettenregel:
		\[
			\dod{\bar\rho}\rho \dod{}{\bar\rho}
			= n \dod{}{\bar\rho}
			\leadsto
			n^2 \dod[2]{}{\bar\rho}
		\]
		Damit folgt:
	}
	\del{n^2 \dod[2]{}{\bar\rho} - n^2 \frac{l(l+1)}{\bar\rho^2} + n \frac{\rho_0}{\bar\rho} - 1} u_l(\rho) &= 0 \\
	\del{\frac{l(l+1)}{\bar\rho^2} - \frac{\rho_0}{n\bar\rho} + \frac{1}{n^2}} u_l(\rho) &= u''_l(\rho)
\end{align*}

Das kommt allerdings nicht hin, da $\rho_0/n \neq 2$ ist.

\subsection{Integration der Differentialgleichung}

\subsubsection{Rechte Seite}

\begin{align*}
	\text{RS}
	&= \del{\frac 1{n^2} - \frac{2}{\bar\rho} + \frac{l(l+1)}{\bar\rho^2}} u_l^2 \del{\bar\rho} \bar\rho^q \\
	&\leadsto \int_0^\infty \dif{\bar\rho} \, \del{\frac 1{n^2} - \frac{2}{\bar\rho} + \frac{l(l+1)}{\bar\rho^2}} u_l^2 \del{\bar\rho} \bar\rho^q \\
	\intertext{%
		Mit der Definition
		\[
			\bracket q := \int_0^\infty \dif{\bar\rho} \,
			u_l^2 \del{\bar\rho} \bar\rho^q
		\]
		folgt:
	}
	&= \frac1{n^2} \bracket q - 2 \bracket{q-1} + l(l+1) \bracket{q-2}
\end{align*}

\subsubsection{Linke Seite}

\begin{align*}
	\text{LS}
	&= u_l''(\bar\rho) u_l\del{\bar\rho} \bar\rho^q \\
	&\leadsto \int_0^\infty \dif{\bar\rho} \, u_l''(\bar\rho) u_l\del{\bar\rho} \bar\rho^q \\
\end{align*}


\subsubsection{Einsetzen in Differentialgleichung}

Beginne mit
\begin{align*}
	\frac{\bracket q}n - 2 \bracket{q-1} + l(l+1) \bracket{q-2}
	&= \half q (q-1) \bracket{q-2} + \int_0^\infty \dif \bar\rho \,
	\frac{\bar\rho^{q+1}}{q+1} 2 u''_l(\bar\rho) u'_l(\bar\rho) \\
	\intertext{%
		Setze die Differentialgleichung ein.
	}
	\frac{\bracket q}n - 2 \bracket{q-1} + l(l+1) \bracket{q-2}
	&= \half q (q-1) \bracket{q-2}
	+ \int_0^\infty \dif \bar\rho \,
	\frac{\bar\rho^{q+1}}{q+1} 2 \del{\frac1{n^2} - \frac2{\bar\rho} + \frac{l(l+1)}{\bar\rho^2}} u_l(\bar\rho) u'_l(\bar\rho) \\
\end{align*}

Betrachte nun nur das Integral und zerlege es in Summanden:
\begin{gather*}
	\int_0^\infty \dif \bar\rho \,
	\frac{\bar\rho^{q+1}}{q+1} 2 \del{\frac1{n^2} - \frac2{\bar\rho}
	+ \frac{l(l+1)}{\bar\rho^2}} u_l(\bar\rho) u'_l(\bar\rho) \\
	=
	\int_0^\infty \dif \bar\rho \,
	\frac{\bar\rho^{q+1}}{q+1} 2
	\frac1{n^2}
	u_l(\bar\rho) u'_l(\bar\rho)
	-
	\int_0^\infty \dif \bar\rho \,
	\frac{\bar\rho^{q+1}}{q+1} 2
	\frac2{\bar\rho}
	u_l(\bar\rho) u'_l(\bar\rho)
	+
	\int_0^\infty \dif \bar\rho \,
	\frac{\bar\rho^{q+1}}{q+1} 2
	\frac{l(l+1)}{\bar\rho^2}
	u_l(\bar\rho) u'_l(\bar\rho)
\end{gather*}

Das erste Integral:
\begin{align*}
	\int_0^\infty \dif \bar\rho \,
	\frac{\bar\rho^{q+1}}{q+1} 2
	\frac1{n^2}
	u_l(\bar\rho) u'_l(\bar\rho)
	&= 
\end{align*}

\subsection{Matrixelemente}

\fehlt

%%%%%%%%%%%%%%%%%%%%%%%%%%%%%%%%%%%%%%%%%%%%%%%%%%%%%%%%%%%%%%%%%%%%%%%%%%%%%%%
%                                    Ende                                     %
%%%%%%%%%%%%%%%%%%%%%%%%%%%%%%%%%%%%%%%%%%%%%%%%%%%%%%%%%%%%%%%%%%%%%%%%%%%%%%%

\IfFileExists{\bibliographyfile}{
	%\bibliography{\bibliographyfile}
}{}

\end{document}

% vim: spell spelllang=de
