 % Copyright © 2012-2013 Martin Ueding <dev@martin-ueding.de>
%
\input{header.tex}

\usepackage{cancel}
\usepackage{tikz}

\newcommand{\themodul}{physik421}
\newcommand{\thegruppe}{Gruppe 4 -- Franz Niecknig}
\newcommand{\theuebung}{7}

\ifoot{\footnotesize{Martin Ueding}}
\ihead{\themodul{} -- Übung \theuebung}
\ofoot{\footnotesize{\thegruppe}}

\def\thesection{H \arabic{section}}
\def\thesubsection{\thesection\alph{subsection}}

\setcounter{section}{9}

\title{\themodul{} -- Übung \theuebung}
\subtitle{\thegruppe}
\author{
	Martin Ueding \footnote{\href{mailto:mu@uni-bonn.de}{mu@uni-bonn.de}}
}

\hypersetup{
	pdftitle={\themodul{} - Übung \theuebung},
}

\begin{document}

\maketitle

\begin{center}
	\ccbysadetitle
\end{center}

Vektor $\vec L$, Tensor $\tens L$, Operator $\hat L$. Leere Stellen in
Matrizen sind 0.

%%%%%%%%%%%%%%%%%%%%%%%%%%%%%%%%%%%%%%%%%%%%%%%%%%%%%%%%%%%%%%%%%%%%%%%%%%%%%%%
%                              Spin-1 Operatoren                              %
%%%%%%%%%%%%%%%%%%%%%%%%%%%%%%%%%%%%%%%%%%%%%%%%%%%%%%%%%%%%%%%%%%%%%%%%%%%%%%%

\section{Spin-1 Operatoren}

\subsection{Matrixdarstellung}

Die Elemente der Matrix $\tens L_z$ berechnen sich wie folgt:
\[
	\del{L_z}_{ij} = \braopket i{\hat L_z}j
	\eqnsep
	i, j \in \set{\uparrow, 0, \downarrow}
\]

\newcommand\lz{\begin{pmatrix}
		1 && \\
		  & 0 & \\
		  && -1
\end{pmatrix}}

Aufgrund der Orthogonalität der Basiszustände und dem Eigenwert $m_l$ gilt:
\[
	\tens L_z = \hbar \lz
\]

Nun $\hat L_\pm$:
\begin{align*}
	\del{L_z}_{\uparrow,\uparrow}
	&= \braopket \uparrow{\hat L_\pm}\uparrow = 0
	&& \text{wegen Orthogonalität}
	\displaybreak[0]
	\\
	%
	\del{L_z}_{\uparrow,0}
	&= \braopket \uparrow{\hat L_\pm}0 = \hbar \sqrt 2
	&& \text{nur für $\hat L_+$}
	\displaybreak[0]
	\\
	%
	\del{L_z}_{\uparrow,\downarrow}
	&= \braopket \uparrow{\hat L_\pm}\downarrow = 0
	\displaybreak[0]
	\\
	%
	\del{L_z}_{0,\uparrow}
	&= \braopket 0{\hat L_\pm}\uparrow = \hbar \sqrt 2
	&& \text{nur für $\hat L_-$}
	\displaybreak[0]
	\\
	%
	\del{L_z}_{0,0}
	&= \braopket 0{\hat L_\pm}0 = 0
	\displaybreak[0]
	\\
	%
	\del{L_z}_{0,\downarrow}
	&= \braopket 0{\hat L_\pm}\downarrow = \hbar \sqrt 2
	&& \text{nur für $L_+$}
	\displaybreak[0]
	\\
	%
	\del{L_z}_{\downarrow,\uparrow}
	&= \braopket \downarrow{\hat L_\pm}\uparrow = 0
	\displaybreak[0]
	\\
	%
	\del{L_z}_{\downarrow,0}
	&= \braopket \downarrow{\hat L_\pm}0 = \hbar \sqrt 2
	&& \text{nur für $\hat L_-$}
	\displaybreak[0]
	\\
	%
	\del{L_z}_{\downarrow,\downarrow}
	&= \braopket \downarrow{\hat L_\pm}\downarrow = 0
\end{align*}

\newcommand\lplus{\begin{pmatrix}
		& 1 & \\
		&& 1 \\
		&&
\end{pmatrix}}

\newcommand\lminus{\begin{pmatrix}
		&& \\
		1 && \\
		  & 1 &
\end{pmatrix}}

\newcommand\lx{\begin{pmatrix}
		& 1 & \\
		1 && 1 \\
		  & 1 &
\end{pmatrix}}

\newcommand\ly{\begin{pmatrix}
		& 1 & \\
		-1 && 1 \\
		  & -1 &
\end{pmatrix}}

\[
	\tens L_+ = \hbar \sqrt 2 \lplus
	\eqnsep
	\tens L_- = \hbar \sqrt 2 \lminus
\]

Mit $\hat L_\pm = \hat L_x \pm \ii \hat L_y$ und
\[
	\hat L_x = \frac 12 \del{\hat L_+ + \hat L_-}
	\eqnsep
	\hat L_y = \frac1{2\ii} \del{\hat L_+ - \hat L_-}
\]
folgt:
\[
	\tens L_x = \frac \hbar 2 \lx
	\eqnsep
	\tens L_y = \frac \hbar {2\ii} \ly
\]

\subsection{Kommutatoren in Matrixdarstellung}

\begin{align*}
	\sbr{\tens L_x, \tens L_y}
	&= \frac{\hbar^2}{4\ii} \del{\lx \ly - \ly \lx} \\
	&= \frac{\hbar^2}{4\ii} \begin{pmatrix}
	-2 && \\
	   & 0 & \\
	   && 2
	\end{pmatrix}
	= \ii \hbar \tens L_z
	\displaybreak[0] \\
	%
	\sbr{\tens L_x, \tens L_z}
	&= \frac{\hbar^2}{2} \del{\lx \lz - \lz \lx} \\
	&= \frac{\hbar^2}{2} \begin{pmatrix}
	& -1 & \\
	1 && -1 \\
	   & 1 &
	\end{pmatrix}
	= \ii \hbar \tens L_y
	\displaybreak[0] \\
	%
	\sbr{\tens L_y, \tens L_z}
	&= - \frac{\hbar^2}{2} \del{\ly \lz - \lz \ly} \\
	&= - \frac{\hbar^2}{2} \begin{pmatrix}
	& -1 & \\
	-1 && -1 \\
	   & -1 &
	\end{pmatrix}
	= \ii \hbar \tens L_x
\end{align*}

\subsection{Kommutator mit Hamiltonoperator}

\begin{align*}
	\hat H
	&= - \frac{\hbar^2}{2m} \laplace + V(r) \\
	\intertext{%
		In Kugelkoordinaten kann der Laplaceoperator in Radial- und
		Winkelanteil separiert werden:
	}
	&= - \frac{\hbar^2}{2m} \del{\laplace_r + \frac{1}{r^2} \laplace_{\theta,\phi}} + V(r) \\
	\intertext{%
		Der Drehimpulsoperator ist gerade $\hat L^2 = \hbar^2
		\laplace_{\theta,\phi}$. Trenne so Radial- und Winkelanteil im
		Hamiltonoperator:
	}
	&= - \frac{\hbar^2}{2m} \laplace_r + \frac{1}{r^2} \hat L^2 + V(r)
\end{align*}

Daraus folgen die Kommutatoren:
\begin{align*}
	\sbr{\dpd{}r, \laplace_{\theta,\phi}} = 0
	&\implies
	\sbr{\hat H, \hat L^2} = 0 \\
	\sbr{\dpd{}r, \dpd{}\phi} = 0
	&\implies
	\sbr{\hat H, \hat L_z} = 0
\end{align*}

%%%%%%%%%%%%%%%%%%%%%%%%%%%%%%%%%%%%%%%%%%%%%%%%%%%%%%%%%%%%%%%%%%%%%%%%%%%%%%%
%                               Hantelmolekül                                %
%%%%%%%%%%%%%%%%%%%%%%%%%%%%%%%%%%%%%%%%%%%%%%%%%%%%%%%%%%%%%%%%%%%%%%%%%%%%%%%

\section{Hantelmolekül}

\fehlt

%%%%%%%%%%%%%%%%%%%%%%%%%%%%%%%%%%%%%%%%%%%%%%%%%%%%%%%%%%%%%%%%%%%%%%%%%%%%%%%
%                                    Ende                                     %
%%%%%%%%%%%%%%%%%%%%%%%%%%%%%%%%%%%%%%%%%%%%%%%%%%%%%%%%%%%%%%%%%%%%%%%%%%%%%%%

\IfFileExists{\bibliographyfile}{
	%\bibliography{\bibliographyfile}
}{}

\end{document}

% vim: spell spelllang=de
