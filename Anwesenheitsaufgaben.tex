\input{header.tex}

\hypersetup{
	pdftitle=
}

\title{physik421 Anwesenheitsaufgaben 1}
\author{
	Martin Ueding \\
	\small{\href{mailto:mu@uni-bonn.de}{mu@uni-bonn.de}}
}

\begin{document}

\maketitle

\section{Rückblick -- Hamiltonsche Mechanik}

\subsection{explizite Formel}

\begin{align*}
	\dif H 
	&= \sum_i \del{p_i \dif{\dot q_i} + \dot q_i \dif{p_i} - \dpd L{\dot q_i} \dif{\dot q_i} - \dpd L{p_i} \dif{p_i}} \\
	&= \sum_i \del{p_i \dif{\dot q_i} + \dot q_i \dif{p_i} - p_i \dif{\dot q_i} - \dpd L{p_i} \dif{p_i}} \\
\end{align*}

\subsection{H als Funktion von $p$, $q$ und $t$}

\[
	H = \sum_i \frac{p_i^2}{2m} + V(q) \\
\]

Totales Differential:
\[
	\dif H = \sum_i \frac{p_i}{m} \dif p_i + \dpd Vq \dif q
\]

\subsection{Koeffizientenvergleich}

Es gibt doppelte so viele Gleichungen, allerdings nur erster Ordnung.

%\IfFileExists{\bibliographyfile}{
%	\bibliography{\bibliographyfile}
%	\bibliographystyle{plain}
%}{}

\end{document}

% vim: spell spelllang=de
